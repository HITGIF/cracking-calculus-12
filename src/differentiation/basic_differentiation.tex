\chapterimage{img/tools.jpg} 
\chapter{Basic Differentiation}

In calculus, you'll be asked to do two things: differentiate and integrate. In this chapter, you're going to learn differentiation.

\section{Notation}
As we've talked about, there are several different notations for derivatives in calculus. We'll use two different types interchangably throughout this book.

The derivatives of the functions will use notation that depends on the function, as shown in the following table:

\begin{table}[H]
    \centering
    \begin{tabular}{l l l}
        \toprule
        \textbf{Function} & \textbf{First Derivative} & \textbf{Second Derivative} \\
        \midrule
        $f(x)$      & $f'(x)$           & $f''(x)$\\
        $g(x)$      & $g'(x)$           & $g''(x)$\\
        $y$         & $y'$ or $\dydx$   & $y''$ or $\dfrac{d^2y}{dx^2}$   \\
        \bottomrule
    \end{tabular}
\end{table}

It is tedious to compute a limit every time we need to know the
derivative of a function. Fortunately, we can develop a small
collection of examples and rules that allow us to compute the
derivative of almost any function we are likely to encounter. \cite{mooc} We will start simply and build-up to more complicated examples.

\section{The Constant Rule}

The simplest function is a constant function.  Recall that derivatives measure the rate of change of a function at a given point. Hence, the derivative of a constant function is zero. For example, the constant function plots a horizontal line---so the slope of the tangent line is $0$. \cite{mooc}
This lead us to our next theorem.

\begin{theorem}[The Constant Rule]
    Given a constant $c$,
    $$\ddx c = 0.$$
\end{theorem}

\begin{example}
    Find $\displaystyle\ddx 114514$.\\
    \begin{solution}
        Guess what, it's 0. (Differentiation is easy!)
    \end{solution}
\end{example}

\section{The Power Rule}
The basic technique for taking a derivative of non-constants is called the \B{Power Rule}.

\begin{theorem}[The Power Rule]\label{T:powerrule}
    For any real number $n$, 
    $$\ddx x^n = n x^{n-1}.$$
\end{theorem}

\begin{example}
    Compute $\ddx x^{13}.$ \cite{mooc}\\
    \begin{solution}
    Applying the power rule, we write
    $\ddx x^{13} = 13x^{12}.$
    \end{solution}
\end{example}

Sometimes, it is not as obvious that one should apply the power rule.\\

\begin{example}
Compute$\ddx \frac{1}{x^4}$. \cite{mooc}\\
\begin{solution}
Applying the power rule, we write$\ddx \frac{1}{x^4} = \ddx x^{-4} = -4x^{-5}$.
\end{solution}
\end{example}

The power rule also applies to radicals once we rewrite them as exponents.\\

\begin{example}
Compute$\ddx \sqrt[5]{x}$. \cite{mooc}\\
\begin{solution}
Applying the power rule, we write$\ddx \sqrt[5]{x} = \ddx x^{1/5} = \frac{x^{-4/5}}{5}$.
\end{solution}
\end{example}

\section{The Sum Rule}

The \textit{sum rule} allows us to take derivatives of functions ``one piece at a time''.

\begin{theorem}[The Sum Rule]\label{theorem:sum rule}
If $f(x)$ and $g(x)$ are differentiable and $c$ is a constant, then 
\begin{enumerate}
    \item\label{SR:1} $\ddx \big( f(x) + g(x)\big) = f'(x) + g'(x)$,
    \item $\ddx \big( f(x) - g(x)\big) = f'(x) - g'(x)$,
    \item $\ddx \big(c\cdot f(x)\big) = c\cdot f'(x)$.
\end{enumerate}
\end{theorem}

\begin{example}
    Compute$\ddx \left( x^5+\frac{1}{x}\right).$ \cite{mooc}\\
    
    \begin{solution}
    \begin{align*}
    \ddx \left(x^5+\frac{1}{x}\right) &= \ddx x^5 + \ddx x^{-1} \\
    &=5x^4 - x^{-2}.
    \end{align*}
    \end{solution}
\end{example}
    
    \begin{example}
    Compute$\ddx \left(\frac{3}{\sqrt[3]{x}}-2\sqrt{x}+\frac{1}{x^7}\right).$ \cite{mooc}\\
    
    \begin{solution}
    \begin{align*}
    \ddx \left(\frac{3}{\sqrt[3]{x}}-2\sqrt{x}+\frac{1}{x^7}\right) &= 3\ddx x^{-1/3} -2\ddx x^{1/2}+\ddx x^{-7}\\
    &=-x^{-4/3} - x^{-1/2}-7x^{-8}.
    \end{align*}
    \end{solution}
\end{example}


\section{The Product Rule}

Consider the product of two simple functions, say
$f(x)\cdot g(x)$, where $f(x)=x^2+1$ and $g(x)=x^3-3x$. An obvious guess for the derivative of $f(x)g(x)$ is the product of the derivatives:
\begin{align*}
f'(x)g'(x) &= (2x)(3x^2-3)\\
&= 6x^3-6x.
\end{align*}
Is this guess correct? We can check by rewriting $f(x)$
and $g(x)$ and doing the calculation in a way that is known to
work. Write 
\begin{align*}
f(x)g(x) &= (x^2+1)(x^3-3x)\\
&=x^5-3x^3+x^3-3x\\
&=x^5-2x^3-3x.
\end{align*} 
Hence
\[
\ddx f(x) g(x) = 5x^4-6x^2-3, 
\]
so we see that 
\[
\ddx f(x) g(x) \ne  f'(x)g'(x).
\]
So the derivative of $f(x)g(x)$ is \textbf{not} as simple as
$f'(x)g'(x)$. Never fear, we have a rule for exactly this
situation. \cite{mooc}
\begin{theorem}[The Product Rule]\index{derivative rules!product}\index{product rule}\label{theorem:product-rule}
If $f(x)$ and $g(x)$ are differentiable, then
$$\ddx f(x)g(x) = f(x)g'(x)+f'(x)g(x).$$
\end{theorem}
\clearpage
\begin{example} 
    Let $f(x)=(x^2+1)$ and $g(x)=(x^3-3x)$. Compute:
    $\ddx f(x)g(x).$ \cite{mooc}
    \begin{solution}
    \begin{align*}
    \ddx f(x)g(x) &= f(x)g'(x) + f'(x)g(x)\\
    &=(x^2+1)(3x^2-3) + 2x(x^3-3x).
    \end{align*}
    We could stop here---or expand it if you're asked to
    \begin{align*}
    (x^2+1)(3x^2-3) + 2x(x^3-3x) &= 3x^4-3x^2 +3x^2 -3 + 2x^4-6x^2\\
    &=5x^4-6x^2-3,
    \end{align*}
    \end{solution}
\end{example}

\section{The Quotient Rule}

We'd like to have a formula to compute
$$
\ddx \frac{f(x)}{g(x)}
$$
if we already know $f'(x)$ and $g'(x)$. Instead of attacking this
problem head-on, let's notice that we've already done part of the
problem: $f(x)/g(x)= f(x)\cdot(1/g(x))$, that is, this is really a
product, and we can compute the derivative if we know $f'(x)$ and
$(1/g(x))'$. This brings us to our next derivative rule. \cite{mooc}

\begin{theorem}[The Quotient Rule]\label{theorem:quotient-rule}
If $f(x)$ and $g(x)$ are differentiable, then
$$\ddx \frac{f(x)}{g(x)} = \frac{f'(x)g(x)-f(x)g'(x)}{g(x)^2}.$$
\end{theorem}

A great way to remember this (how I memorized this) is to say:
$$\dfrac{``LoDeHi-HiDeLo"}{(Lo)^2}$$

\begin{example}
    Compute: $\ddx \frac{x^2+1}{x^3-3x}.$ \cite{mooc}
    
    \begin{solution}
    \begin{align*}
    \ddx \frac{x^2+1}{x^3-3x} &= \frac{2x(x^3-3x)-(x^2+1)(3x^2-3)}{(x^3-3x)^2}\\
    &=\frac{-x^4-6x^2+3}{(x^3-3x)^2}.
    \end{align*}
    \end{solution}
\end{example}
    
It is often possible to calculate derivatives in more than one way, as we have already seen. Since every quotient can be written as a product, it is always possible to use the product rule to compute the derivative, though it is not always simpler. \cite{mooc}
    
\begin{example}
    Compute $$\ddx \frac{625-x^2}{\sqrt{x}}$$
    in two ways. First using the quotient rule and then using the product rule. \cite{mooc}
    
    \begin{solution} 
    First, we'll compute the derivative using the quotient rule. 
    \[
    \ddx \frac{625-x^2}{\sqrt{x}} = \frac{\left(-2x\right)\left(\sqrt{x}\right) - (625-x^2)\left(\frac{1}{2}x^{-1/2}\right)}{x}.
    \]
    Second, we'll compute the derivative using the product rule:
    \begin{align*}
    \ddx \frac{625-x^2}{\sqrt{x}} &= \ddx \left(625-x^2\right)x^{-1/2}\\
    &=\left(625-x^2\right)\left(\frac{-x^{-3/2}}{2}\right)+ (-2x)\left(x^{-1/2}\right).
    \end{align*}
    With a bit of algebra, both of these simplify to
    $$
    -\frac{3x^2+625}{2x^{3/2}}.
    $$
    \end{solution}
\end{example}

\section{The Chain Rule}

Consider
$$
h(x) = (1+2x)^5.
$$

While there are several different ways to differentiate this function, if we let $f(x) = x^5$ and $g(x) = 1+2x$, then we can express $h(x) =f(g(x))$. The question is, can we compute the derivative of a composition of functions using the derivatives of the constituents $f(x)$ and $g(x)$? To do so, we need the \textit{chain rule}. \cite{mooc}

\begin{theorem}[Chain Rule]
    If $f(x)$ and $g(x)$ are differentiable, then
    $$
    \ddx f(g(x)) = f'(g(x))g'(x).
    $$
\end{theorem}

And the last bits of examples.\\

\begin{example}
    Compute: $\ddx (1+2x)^5$. \cite{mooc}
    \begin{solution}
    Set $f(x) = x^5$ and $g(x) = 1+2x$, now
    \[
    f'(x) = 5x^4 \qquad\text{and}\qquad g'(x) = 2.
    \]
    Hence
    \begin{align*}
    \ddx (1+2x)^5 &= \ddx f(g(x))\\ 
    &=f'(g(x))g'(x) \\
    &= 5(1+2x)^4\cdot 2\\
    &= 10(1+2x)^4.
    \end{align*}
    \end{solution}
\end{example}
\clearpage
\begin{example}
    Compute: $\ddx \sqrt{1+\sqrt{x}}$. \cite{mooc}
    
    \begin{solution}
    Set 
    $f(x)=\sqrt{x}$ and $g(x)=1+x$. Hence,
    \[
    \sqrt{1+\sqrt{x}}=f(g(f(x))) \qquad\text{and}\qquad\ddx f(g(f(x))) = f'(g(f(x)))g'(f(x))f'(x).
    \]
    Since 
    \[
    f'(x) = \frac{1}{2\sqrt{x}} \qquad\text{and}\qquad g'(x) = 1
    \]
    We have that
    \[
    \ddx \sqrt{1+\sqrt{x}} = \frac{1}{2\sqrt{1+\sqrt{x}}}\cdot 1\cdot  \frac{1}{2\sqrt{x}}.
    \]
    \end{solution}
\end{example}
~\\
\begin{exercise}
    ~\\\\\-\hspace{0.3cm} \textbf{
        In Exercises 1–18, find the $derivative$.
    }\cite{mooc}
    \twocol
    \begin{enumerate} 
        \item $\ddx 2147483647$
        \item $\ddx \frac{1}{\sqrt{2}}$
        
        \item $\ddx x^\pi$
        \item $\ddx \frac{1}{(\sqrt[7]{x})^9}$
        
        \item $\ddx \left(5x^3+12x^2-15\right)$
        \item $\ddx \left(\frac{x^2}{x^7}+\frac{\sqrt{x}}{x}\right)$
        
        \item $\ddx x^3(x^3-5x+10)$
        \item $\ddx (x^2+5x-3)(x^5-6x^3+3x^2-7x+1)$
        
		\item $\D \ddx {x^3\over x^3-5x+10}$
		\item $\D \ddx {x^2+5x-3\over x^5-6x^3+3x^2-7x+1}$

		\item $\D \ddx (1+3x)^2$
		\item $\D \ddx \sqrt{{169\over x}-x}$
		\item $\D \ddx 100/(100-x^2)^{3/2}$
		\item $\D \ddx \sqrt{(x^2+1)^2+\sqrt{1+(x^2+1)^2}}$
		\item $\D \ddx (3x^2+1)(2x-4)^3$
		\item $\D \ddx {2x^{-1}-x^{-2}\over 3x^{-1}-4x^{-2}}$
		\item $\D \ddx (2x+1)^3(x^2+1)^2$
		\item $\D \ddx x^4-3x^3+(1/2)x^2+7x-\pi$
    \end{enumerate}
    \endtwocol
\end{exercise}
