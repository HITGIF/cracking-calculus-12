\chapterimage{img/implicit.jpg} 
\chapter{Implicit Differentiation}

\section{How To Do It}

The functions we've been dealing with so far have been
\textit{explicit functions}\index{explicit function}, meaning that the
dependent variable is written in terms of the independent variable. \cite{mooc} For
example:
$$
y=3x^2-2x+1,\qquad y=e^{3x}, \qquad y = \frac{x-2}{x^2-3x+2}.
$$
However, there are another type of functions, called \textit{implicit
  functions}. In this case, the dependent variable is not stated
explicitly in terms of the independent variable. \cite{mooc} For example:
$$
x^2+y^2 = 4,\qquad x^3+y^3 = 9xy, \qquad x^4+3x^2 = x^{2/3}+y^{2/3} = 1.
$$
Your inclination might be simply to solve each of these for $y$ and go
merrily on your way. However this can be difficult and it may require
two \textit{branches}, for example to explicitly plot $x^2+y^2 = 4$,
one needs both $y= \sqrt{4-x^2}$ and $y=-\sqrt{4-x^2}$. Moreover, it
may not even be possible to solve for $y$. To deal with such
situations, we use \index{implicit differentiation}\textit{implicit
  differentiation}. \cite{mooc} Let's see an illustrative example:

  \begin{example}
    Consider the curve defined by
    $$
    x^3+y^3 = 9xy.
    $$ 
    \begin{figure}[H]
        \centering
        \begin{tikzpicture}
            \begin{axis}[
                    xmin=-6,xmax=6,ymin=-6,ymax=6,
                    height=6cm,
                    axis lines=center,
                    xlabel=$x$, ylabel=$y$,
                    every axis y label/.style={at=(current axis.above origin),anchor=south},
                    every axis x label/.style={at=(current axis.right of origin),anchor=west},
                  ]        
                  \addplot [very thick, penColor, smooth, samples=100, domain=(-.99:0)] ({9*x/(1+x^3)},{9*x^2/(1+x^3)});
                  \addplot [very thick, penColor, smooth, samples=100, domain=(-.99:0)] ({9*x^2/(1+x^3)},{9*x/(1+x^3)});
                  \addplot [very thick, penColor, smooth, samples=100, domain=(0:1)] ({9*x/(1+x^3)},{9*x^2/(1+x^3)});
                  \addplot [very thick, penColor, smooth, samples=100, domain=(0:1)] ({9*x^2/(1+x^3)},{9*x/(1+x^3)});
                \end{axis}
        \end{tikzpicture}
        \label{plot:x^3+y^3=9xy}
        \caption{A plot of $x^3+y^3 = 9xy$. \cite{mooc}}
    \end{figure}
    \begin{enumerate}
    \item Compute $\dydx$. \cite{mooc}
    \item Find the slope of the tangent line at $(4,2)$. \cite{mooc}
    \end{enumerate}
~\\
    \begin{solution}
        Starting with $x^3+y^3 = 9xy$, we apply the differential operator $\ddx$ to both sides of the
        equation to obtain
        \[
        \ddx \left(x^3+y^3\right) = \ddx 9xy.
        \]
        Applying the sum rule we see
        \[
        \ddx x^3+\ddx y^3 = \ddx 9xy.
        \]
        Let's examine each of these terms in turn. To start
        \[
        \ddx x^3 = 3x^2.
        \]
        On the other hand $\ddx y^3$ is somewhat different. Here you imagine that $y = y(x)$, and hence by the chain rule
        \begin{align*}
        \ddx y^3 &= \ddx (y(x))^3 \\ 
        &= 3(y(x))^2 \cdot y'(x) \\
        &= 3y^2\dydx.
        \end{align*}
        Considering the final term $\ddx 9xy$, we again imagine that $y=y(x)$. Hence 
        \begin{align*}
        \ddx 9xy &= 9\ddx x\cdot y(x) \\
        &= 9 \left(x\cdot y'(x) + y(x)\right)\\
        &= 9x \dydx + 9y.
        \end{align*}
        Putting this all together we are left with the equation
        \[
        3x^2 + 3y^2\dydx =9x \dydx + 9y.
        \]
        At this point, we solve for $\dydx$. Write
        \begin{align*}
        3x^2 + 3y^2\dydx &= 9x \dydx + 9y\\
        3y^2\dydx -  9x \dydx &= 9y - 3x^2\\
        \dydx\left(3y^2-9x\right)&= 9y - 3x^2\\
        \dydx &=\frac{9y - 3x^2}{3y^2-9x} = \frac{3y - x^2}{y^2-3x}.
        \end{align*}
        
        For the second part of the problem, we simply plug $x=4$ and $y=2$
        into the formula above, hence the slope of the tangent line at $(4,2)$
        is $\D\frac{5}{4}$, see Figure~\ref{plot:x^3+y^3=9xy}.
    \end{solution}
\end{example}

\section{Second Derivatives}
Sometimes, you'll be asked to find a second derivative implicitly.

\begin{example}
    Find $\ddydx$ if $y^2+2y=4x^2+2x$. \cite{ap}\\
    \begin{solution}
        Differentiating implicitly, you get
        \[2y\dydx+2\dydx=8x+2\]
        Now, simplify and solve for $\dydx$.
        \[\dydx=\dfrac{4x+1}{y+1}\]
        Noew, it's time to take the derivative again.
        \[\ddydx=\dfrac{4(y+1)-(4x+1)\left(\dydx\right)}{(y+1)^2}\]
        What's $\dydx$? Well, you just defined it yourself.
        \begin{align*}
            \ddydx
            &=\dfrac{4(y+1)-(4x+1)\left(\dfrac{4x+1}{y+1}\right)}{(y+1)^2}\\
            &=\dfrac{4(y+1)^2-(4x+1)^2}{(y+1)^3}
        \end{align*}
    \end{solution}
\end{example}

That's how you do implicit differntiation. Give yourself a rest before starting these exercises.

\begin{exercise}
    ~\\\\\-\hspace{0.3cm} \textbf{
        In Exercises 1–6, find $\dydx$.
    }\cite{mooc}
    \twocol
    \begin{enumerate} 
        \item $x^2 + y^2 = 4$
        \item $x^2+xy+y^2=7$
        \item $x^3+xy^2=y^3+yx^2$
        \item $\sqrt{x} + \sqrt{y} = 9$
        \item $xy^{3/2}+4 = 2x+y$
        \item ${1\over x} + {1\over y} = 7$
    \end{enumerate}
    \endtwocol
    ~\\\\\-\hspace{0.3cm} \textbf{
        In Exercises 7–8, find $\ddydx$.
    }\cite{ci}
    \twocol
    \begin{enumerate} 
        \setcounter{enumi}{6}
        \item $x^2+y^2=4$
        \item $y^2=x^3$
    \end{enumerate}
    \endtwocol
\end{exercise}
