\chapterimage{img/mvtd.jpg} 
\chapter{Basic Applications of the Derivative}

Consider this problem: Suppose you drive a car from toll booth on a toll road to another toll booth at an average speed of 70 miles per hour. What can be concluded about your actual speed during the trip? In particular, did you exceed the 65 mile per hour speed limit? To solve this, you need to use the \B{Mean Value Theorem}.

\section{Mean Value Theorem for Derivatives}
\hfil
\begin{theorem} [Mean Value Theorem]\hfil
    \label{thm:mvt}
    Suppose that $f(x)$ has a derivative on the
    interval $(a,b)$ and is continuous on the interval $[a,b]$. 
    Then at some value $c\in (a,b)$,
    $$f'(c)={f(b)-f(a)\over b-a}.\text{ \cite{mooc}}$$
\end{theorem}

In other words, there's come point in the interval where the slope of the tangent line equals the slope of the secant line that connects the endpoints of the interval. You can see this in Figure \ref{figure:geoMVT}:

\begin{figure}[H]
    \centering
    \begin{tikzpicture}
        \begin{axis}[
                xmin=.5, xmax=5.5,ymin=0,ymax=3.1,
                height=5cm,
                axis lines =center, xlabel=$x$, ylabel=$y$,
                every axis y label/.style={at=(current axis.above origin),anchor=south},
                every axis x label/.style={at=(current axis.right of origin),anchor=west},
                xtick={1,2.04,5}, xticklabels={$a$,$c$,$b$},
                ytickmin=1, ytickmax=0,
                axis on top,
              ] 
              \addplot [draw=none, fill=fill2,domain=(1:5)] {3.1} \closedcycle;       
              \addplot [penColor2!40!background,very thick,dashed] plot coordinates {(1,.84+1.5) (5,1.5-.96)};        
              \addplot [textColor,dashed] plot coordinates {(2.04,0) (2.04,1.5+.89)};        
          \addplot [very thick,penColor, smooth,domain=(1:5)] {sin(deg(x))+1.5};
          \addplot [very thick,penColor2,domain=(.5:5.5)] {-.45*(x-2.04)+.89+1.5};
          \addplot [very thick,penColor2,domain=(.5:5.5)] coordinates {(1,.84+1.5) (5,1.5-.96)};
              \node at (axis cs:4,.4) [penColor] {$f(x)$}; 
              \addplot[color=penColor,fill=penColor,only marks,mark=*] coordinates{(1,.84+1.5)};  %% closed hole    
              \addplot[color=penColor,fill=penColor,only marks,mark=*] coordinates{(5,-.96+1.5)};  %% closed hole   
              \addplot[color=penColor2,fill=penColor2,only marks,mark=*] coordinates{(2.04,.89+1.5)};  %% closed hole          
            \end{axis}
    \end{tikzpicture}
    \caption{A geometric interpretation of the Mean Value Theorem}
    \label{figure:geoMVT}
\end{figure}

\begin{example}
    Suppose you drive a car from toll booth on a toll road to another toll
    booth $30$ miles away in half of an hour. Must you have been driving
    at $60$ miles per hour at some point? \cite{mooc}\\
    
    \begin{solution}
    If $p(t)$ is the position of the car at time $t$, and $0$ hours is
    the starting time with $1/2$ hours being the final time, the Mean Value Theorem states there is a time $c$
    \[
    p'(c) = \frac{30-0}{1/2} = 60\qquad \text{where $0<c<1/2$.}
    \]
    Since the derivative of position is velocity, this says that the car
    must have been driving at $60$ miles per hour at some point.
    \end{solution}
\end{example}

\section{Rolle's Theorem}
Now let's learn Rolle's Theorem, which is a special case of the MVTD.

\begin{theorem}[Rolle's Theorem]\index{Rolle's Theorem} 
    Suppose that $f(x)$ is differentiable on the interval $(a,b)$, is
    continuous on the interval $[a,b]$, and $f(a)=f(b)$. Then 
    \[
    f'(c)=0
    \]
    for some $a<c<b$.
    \label{thm:rolle}
\end{theorem}
\begin{figure}[H]
    \centering
    \begin{tikzpicture}
        \begin{axis}[
                xmin=0, xmax=4.5,ymin=1,ymax=5,
                axis lines =left, xlabel=$x$, ylabel=$y$,
                every axis y label/.style={at=(current axis.above origin),anchor=south},
                every axis x label/.style={at=(current axis.right of origin),anchor=west},
                xtick={1,2,3}, xticklabels={$a$,$c$,$b$},
                ytickmin=1, ytickmax=0,
                axis on top,
              ]       
              \addplot [draw=none, fill=fill2,domain=(1:3)] {5} \closedcycle;       
          \addplot [very thick,penColor, smooth] {-(x-2)^2+4};
              \addplot [very thick,penColor2, smooth] {4};
              \node at (axis cs:.4,2.5) [penColor] {$f(x)$}; 
              \addplot [textColor,dashed] plot coordinates {(2,0) (2,4)};
              \addplot [textColor,dashed] plot coordinates {(1,3) (3,3)};
              \addplot[color=penColor2,fill=penColor2,only marks,mark=*] coordinates{(2,4)};  %% closed hole          
              \addplot[color=penColor,fill=penColor,only marks,mark=*] coordinates{(1,3)};  %% closed hole          
              \addplot[color=penColor,fill=penColor,only marks,mark=*] coordinates{(3,3)};  %% closed hole          
            \end{axis}
    \end{tikzpicture}
    \caption{A geometric interpretation of Rolle's Theorem.}
    \label{figure:geoRolle}
\end{figure}

\begin{example}
    Suppose you toss a ball into the air and then catch it. Must the ball's vertical velocity have been zero at some point? \cite{mooc}\\
    \begin{solution}
    If $p(t)$ is the position of the ball at time $t$, then we may apply Rolle's Theorem to see at some time $c$, $p'(c)=0$. Hence the velocity must be zero at some point.
    \end{solution}
\end{example}
\clearpage

\begin{exercise}
    ~\\\\\-\hspace{0.3cm} \textbf{
        In Exercises 1–3, determine whether Rolle’s Theorem can be applied to $f$ on the closed interval $[a, b]$. If Rolle’s Theorem can be applied, find all values of $c$ in the open interval $(a, b)$ such that $f'(c)=0$. If Rolle’s Theorem cannot be applied, explain why not.
    }\cite{ci}\\
    \begin{enumerate} 
        \item $f(x)=(x-1)(x-2)(x-3), \quad [1,3]$
        \item $f(x)=x^{2/3}-1, \quad [-8,8]$
        \item $f(x)=\dfrac{x^2-2x-3}{x+2}, \quad [-1,3]$
    \end{enumerate}
    ~\\\\\-\hspace{0.3cm} \textbf{
        In Exercises 4–6, determine whether Mean Value Theorem can be applied to $f$ on the closed interval $[a, b]$. If Mean Value Theorem can be applied, find all values of $c$ in the open interval $(a, b)$ such that $f'(c)=\dfrac{f(b)-f(a)}{b-a}$. If Mean Value Theorem cannot be applied, explain why not.
    }\cite{ci}\\
    \begin{enumerate} 
        \setcounter{enumi}{3}
        \item $f(x)=x^2, \quad [-2,1]$
        \item $f(x)=x^3+2x, \quad [-1,1]$
        \item $f(x)=|2x+1|, \quad [-1,3]$
    \end{enumerate}
\end{exercise}
