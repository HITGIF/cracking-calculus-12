\chapterimage{img/curve.jpg} 
\chapter{Curve Sketching}

\section{The First Derivative Test}

The method of the previous section for deciding whether there is a
relative maximum or minimum at a critical point by testing ``near-by''
points is not always convenient. Instead, since we have already had to
compute the derivative to find the critical points, we can use
information about the derivative to decide. \cite{mooc} Recall that
~\\
\begin{itemize}
\item If $f'(x) >0$ on an interval, then $f(x)$ is increasing on that interval.
\item If $f'(x) <0$ on an interval, then $f(x)$ is decreasing on that interval.
\end{itemize}
~\\
So how exactly does the derivative tell us whether there is a maximum,
minimum, or neither at a point? Use the \textit{first derivative test}. \cite{mooc}

\begin{theorem}[First Derivative Test]
    \label{T:fdt}\hfil
    Suppose that $f(x)$ is continuous on an interval, and that $f'(a)=0$
    for some value of $a$ in that interval.
    \begin{itemize}
        \item If $f'(x)>0$ to the left of $a$ and $f'(x)<0$ to the right of
        $a$, then $f(a)$ is a relative maximum.
        \item If $f'(x)<0$ to the left of $a$ and $f'(x)>0$ to the right of
        $a$, then $f(a)$ is a relative minimum.
        \item If $f'(x)$ has the same sign to the left and right of $a$,
        then $f(a)$ is not a relative extremum.
    \end{itemize}
    \cite{mooc}
\end{theorem}

\hfil
\begin{example}\label{E:relativeextrema}
    Consider the function 
    $$
    f(x) = \frac{x^4}{4}+\frac{x^3}{3}-x^2
    $$
    Find the intervals on which $f(x)$ is increasing and decreasing and
    identify the relative extrema of $f(x)$. \cite{mooc}
    \clearpage
    \begin{solution}
    Start by computing
    \[
    \ddx f(x) = x^3+x^2-2x.
    \]
    Now we need to find when this function is positive and when it is
    negative. To do this, solve 
    \[
    f'(x) = x^3+x^2-2x =0.
    \]
    Factor $f'(x)$
    \begin{align*}
    f'(x) &= x^3+x^2-2x \\
    &=x(x^2+x-2)\\
    &=x(x+2)(x-1).
    \end{align*}
    So the critical points (when $f'(x)=0$) are when $x=-2$, $x=0$, and
    $x=1$. Now we can check points \textbf{between} the critical points to find
    when $f'(x)$ is increasing and decreasing:
    \[
    f'(-3)=-12 \qquad f'(.5)=-0.625 \qquad f'(-1)=2 \qquad f'(2)=8
    \]
    From this we can make a sign table:
    
    \flushleft
    \begin{tikzpicture}
        \begin{axis}[
                trim axis left,
                scale only axis,
                domain=-3:3,
                ymax=2,
                ymin=-2,
                axis lines=none,
                height=3cm, %% Hard coded height! 
                width=\textwidth, %% width
              ]
              \addplot [draw=none, fill=fill1, domain=(-3:-2)] {2} \closedcycle;
              \addplot [draw=none, fill=fill2, domain=(-2:0)] {2} \closedcycle;
              \addplot [draw=none, fill=fill1, domain=(0:1)] {2} \closedcycle;
              \addplot [draw=none, fill=fill2, domain=(1:3)] {2} \closedcycle;
              
              \addplot [->,textColor] plot coordinates {(-3,0) (3,0)}; %% axis{0};
              
              \addplot [dashed, textColor] plot coordinates {(-2,0) (-2,2)};
              \addplot [dashed, textColor] plot coordinates {(0,0) (0,2)};
              \addplot [dashed, textColor] plot coordinates {(1,0) (1,2)};
              
              \node at (axis cs:-2,0) [anchor=north,textColor] {\footnotesize$-2$};
              \node at (axis cs:0,0) [anchor=north,textColor] {\footnotesize$0$};
              \node at (axis cs:1,0) [anchor=north,textColor] {\footnotesize$1$};
    
              \node at (axis cs:-2.5,1) [textColor] {\footnotesize$f'(x)<0$};
              \node at (axis cs:.5,1) [textColor] {\footnotesize$f'(x)<0$};
              \node at (axis cs:-1,1) [textColor] {\footnotesize$f'(x)>0$};
              \node at (axis cs:2,1) [textColor] {\footnotesize$f'(x)>0$};
    
              \node at (axis cs:-2.5,-.5) [anchor=north,textColor] {\footnotesize Decreasing};
              \node at (axis cs:.5,-.5) [anchor=north,textColor] {\footnotesize Decreasing};
              \node at (axis cs:-1,-.5) [anchor=north,textColor] {\footnotesize Increasing};
              \node at (axis cs:2,-.5) [anchor=north,textColor] {\footnotesize Increasing};
    
            \end{axis}
    \end{tikzpicture}
    
    Hence $f(x)$ is increasing on $(-2,0)\cup(1,\infty)$ and $f(x)$ is
    decreasing on $(-\infty,-2)\cup(0,1)$. Moreover, from the first
    derivative test, Theorem~\ref{T:fdt}, the relative maximum is at $x=0$
    while the relative minima are at $x=-2$ and $x=1$, see
    Figure~\ref{figure:(x^4)/4 + (x^3)/3 -x^2}.
    \end{solution}
    \begin{figure}[H]
        \centering
        \begin{tikzpicture}
            \begin{axis}[
                    domain=-4:4,
                    ymax=5,
                    ymin=-5,
                    %samples=100,
                    axis lines =middle, xlabel=$x$, ylabel=$y$,
                    every axis y label/.style={at=(current axis.above origin),anchor=south},
                    every axis x label/.style={at=(current axis.right of origin),anchor=west}
                ]
                \addplot [dashed, textColor, smooth] plot coordinates {(-2,0) (-2,-2.667)}; %% {.451};
                \addplot [dashed, textColor, smooth] plot coordinates {(1,0) (1,-.4167)}; %% axis{2.215};
        
                \addplot [very thick, penColor, smooth] {(x^4)/4 + (x^3)/3 -x^2};
                \addplot [very thick, penColor2, smooth] {x^3 + x^2 -2*x};
        
                \node at (axis cs:-1.3,-2) [anchor=west] {\color{penColor}$f(x)$};  
                \node at (axis cs:-2.1,2) [anchor=west] {\color{penColor2}$f'(x)$};
        
                \addplot[color=penColor2,fill=penColor2,only marks,mark=*] coordinates{(-2,0)};  %% closed hole
                \addplot[color=penColor2,fill=penColor2,only marks,mark=*] coordinates{(1,0)};  %% closed hole
                \addplot[color=penColor2,fill=penColor2,only marks,mark=*] coordinates{(0,0)};  %% closed hole
                \addplot[color=penColor,fill=penColor,only marks,mark=*] coordinates{(-2,.-2.667)};  %% closed hole
                \addplot[color=penColor,fill=penColor,only marks,mark=*] coordinates{(1,-.4167)};  %% closed hole
            \end{axis}
        \end{tikzpicture}
        \caption{A plot of $f(x) =x^4/4 + x^3/3 -x^2$ and $f'(x) = x^3 + x^2 -2x$. \cite{mooc}}
        \label{figure:(x^4)/4 + (x^3)/3 -x^2}
    \end{figure}
\end{example}

Hence we have seen that if $f'(x)$ is zero and increasing at a point,
then $f(x)$ has a relative minimum at the point. If $f'(x)$ is zero and
decreasing at a point then $f(x)$ has a relative maximum at the
point. Thus, we see that we can gain information about $f(x)$ by
studying how $f'(x)$ changes. This leads us to our next section. \cite{mooc}




\section{Concavity and Inflection Points}

We know that the sign of the derivative tells us whether a function is
increasing or decreasing. Likewise, the sign of the second derivative
$f''(x)$ tells us whether $f'(x)$ is increasing or decreasing. We summarize this in the table below: \cite{mooc}\\

{\setlength{\arrayrulewidth}{5pt}
\taburulecolor{textColor!10!background}
\begin{tabu}{c|c|c|} %% gives thick lines
 & $f'(x)<0$ & $f'(x) > 0$ \\ \hline & & \\[-1.5ex]
$f''(x)> 0$ & 
\begin{minipage}{2in}
\[
\begin{tikzpicture}
	\begin{axis}[
            clip=false,
            height=4.5cm,
            domain=0:1,
            ymax=1,
            ymin=0,
            axis lines=none,
          ]
          \addplot [very thick, penColor, smooth] {(x-1)^2};
          \node at (axis cs:.7,.4) [textColor] {\footnotesize Concave Up};
        \end{axis}
\end{tikzpicture}
\]
\begin{minipage}{2in}\footnotesize
Here $f'(x)<0$ and $f''(x)>0$. This means that $f(x)$ slopes down and
is getting \textit{less steep}. In this case the curve is
\textbf{concave up}.
\end{minipage}
\end{minipage}
&
\begin{minipage}{2in}
\[
\begin{tikzpicture}
	\begin{axis}[
            clip=false,
            domain=0:1,
            ymax=1,
            height=4.5cm,
            ymin=0,
            axis lines=none,
          ]
          \addplot [very thick, penColor, smooth] {x^2};
          \node at (axis cs:.3,.4) [textColor] {\footnotesize Concave Up};
        \end{axis}
\end{tikzpicture}
\]
\begin{minipage}{2in}\footnotesize
Here $f'(x)>0$ and $f''(x)>0$. This means that $f(x)$ slopes up and is
getting \textit{steeper}. In this case the curve is \textbf{concave
  up}.
\end{minipage}
\end{minipage}
\\[-2ex]
& & 
\\\hline 
& & \\[-1.5ex]
$f''(x)<0$ &
\begin{minipage}{2in}
\[
\begin{tikzpicture}
	\begin{axis}[
            clip=false,
            height=4.5cm,
            domain=0:1,
            ymax=1,
            ymin=0,
            axis lines=none,
          ]
          \addplot [very thick, penColor, smooth] {-x^2+1};
          \node at (axis cs:.4,.4) [textColor] {\footnotesize Concave Down};
        \end{axis}
\end{tikzpicture}
\]
\begin{minipage}{2in}\footnotesize
Here $f'(x)<0$ and $f''(x)<0$. This means
that $f(x)$ slopes down and is getting \textit{steeper}. In this case the curve is \textbf{concave down}.
\end{minipage}
\end{minipage}
&
\begin{minipage}{2in}
\[
  \begin{tikzpicture}
	\begin{axis}[
            clip=false,
            height=4.5cm,
            domain=0:1,
            ymax=1,
            ymin=0,
            axis lines=none,
          ]
          \addplot [very thick, penColor, smooth] {-(x-1)^2+1};
          \node at (axis cs:.6,.4) [textColor] {\footnotesize Concave Down};
        \end{axis}
\end{tikzpicture}
\]
\begin{minipage}{2in}\footnotesize
Here $f'(x)>0$ and $f''(x)<0$. This means
that $f(x)$ slopes up and is getting less \textit{steep}. In this case the curve is \textbf{concave down}.
\end{minipage}
\end{minipage}
\\[-2ex]
& & 
\\\hline 
\end{tabu}}
    

If we are trying to understand the shape of the graph of a function,
knowing where it is concave up and concave down helps us to get a more
accurate picture. It is worth summarizing what we have seen already in
to a single theorem.

\begin{theorem}[Test for Concavity]\index{concavity test}
Suppose that $f''(x)$ exists on an interval.
\begin{enumerate}
\item If $f''(x)>0$ on an interval, then $f(x)$ is concave up on that interval.
\item If $f''(x)<0$ on an interval, then $f(x)$ is concave down on that interval.
\end{enumerate}
\end{theorem}


Of particular interest are points at which the concavity changes from
up to down or down to up. 

\begin{definition}\index{inflection point}
If $f(x)$ is continuous and its concavity changes either from up to
down or down to up at $x=a$, then $f(x)$ has an \textbf{inflection
  point} at $x=a$.
\end{definition}

It is instructive to see some examples and nonexamples of inflection
points.

\begin{tabular}{cccc}
\begin{tikzpicture}
	\begin{axis}[
            domain=0:2,
            ymax=2,
            height=4.5cm,
            ymin=0,
            axis lines=none,
          ]
          \addplot [very thick, penColor, smooth, domain=(0:1)] {(x-1)^2+1};
          \addplot [very thick, penColor, smooth, domain=(1:2)] {-(x-1)^2+1};
          \addplot[color=penColor,fill=penColor,only marks,mark=*] coordinates{(1,1)};
        \end{axis}
\end{tikzpicture}

&

\begin{tikzpicture}
	\begin{axis}[
            height=4.5cm,
            domain=0:2,
            ymax=1,
            ymin=0,
            axis lines=none,
          ]
          \addplot [very thick, penColor2, smooth] {-(x-1)^2+.75};
          \addplot[color=penColor2,fill=penColor2,only marks,mark=*] coordinates{(1,.75)};
        \end{axis}
\end{tikzpicture} 

\\[-2ex]
& & 
\\\hline 
& & \\[-1.5ex]

\begin{minipage}{2in}\footnotesize
    This is an inflection point. The concavity changes from concave up to
    concave down.
\end{minipage}
    
& 
    
\begin{minipage}{2in}\footnotesize
    This is \textbf{not} an inflection point. The curve is concave down on either side of the point.
\end{minipage}
\\
\begin{tikzpicture}
	\begin{axis}[
            height=4.5cm,
            domain=0:2,
            ymax=2,
            ymin=0,
            samples=100,
            axis lines=none,
          ]
          \addplot [very thick, penColor, smooth,domain=(1:2)] {sqrt(x-1)+1};
          \addplot [very thick, penColor, smooth,domain=(0:1)] {-sqrt(abs(1-x))+1};
          \addplot[color=penColor,fill=penColor,only marks,mark=*] coordinates{(1,1)};
        \end{axis}
\end{tikzpicture}

&

\begin{tikzpicture}
	\begin{axis}[
            height=4.5cm,
            domain=0:2,
            ymax=2,
            ymin=0,
            axis lines=none,
          ]
          \addplot [very thick, penColor2, smooth,domain=(1:2)] {sqrt(x-1)+.5};
          \addplot [very thick, penColor2, smooth,domain=(0:1)] {sqrt(abs(1-x))+.5};
          \addplot[color=penColor2,fill=penColor2,only marks,mark=*] coordinates{(1,.5)};
        \end{axis}
\end{tikzpicture} \\

\begin{minipage}{2in}\footnotesize
This is an inflection point. The concavity changes from concave up to concave down.
\end{minipage}

&

\begin{minipage}{2in}\footnotesize
This is \textbf{not} an inflection point. The curve is concave down on either side of the point.
\end{minipage}
\end{tabular}
~\\\\\\\\
We identify inflection points by first finding where $f''(x)$ is zero or undefined and then checking to see whether $f''(x)$ does in fact go from positive to negative or negative to positive at these points.


\section{The Second Derivative Test}


Recall the first derivative test, Theorem~\ref{T:fdt}:\\
\begin{itemize}
\item If $f'(x)>0$ to the left of $a$ and $f'(x)<0$ to the right of
  $a$, then $f(a)$ is a relative maximum.
\item If $f'(x)<0$ to the left of $a$ and $f'(x)>0$ to the right of
  $a$, then $f(a)$ is a relative minimum.
\end{itemize}
~\\
If $f'(x)$ changes from positive to negative it is decreasing. In this
case, $f''(x)$ might be negative, and if in fact $f''(x)$ is negative
then $f'(x)$ is definitely decreasing, so there is a relative maximum at
the point in question. On the other hand, if $f'(x)$ changes from
negative to positive it is increasing. Again, this means that
$f''(x)$ might be positive, and if in fact $f''(x)$ is positive then
$f'(x)$ is definitely increasing, so there is a relative minimum at the
point in question. We summarize this as the \textit{second derivative test}. \cite{mooc}

\begin{theorem}[Second Derivative Test]\index{second derivative test}\label{T:sdt}
Suppose that $f''(x)$ is continuous on an open interval and that
$f'(a)=0$ for some value of $a$ in that interval.
\begin{itemize}
\item If $f''(a) <0$, then $f(x)$ has a relative maximum at $a$.
\item If $f''(a) >0$, then $f(x)$ has a relative minimum at $a$.
\item If $f''(a) =0$, then the test is inconclusive. In this case,
  $f(x)$ may or may not have a relative extremum at $x=a$.
\end{itemize}
\end{theorem}


The second derivative test is often the easiest way to identify relative
maximum and minimum points. Sometimes the test fails and sometimes
the second derivative is quite difficult to evaluate. In such cases we
must fall back on one of the previous tests. \cite{mooc}

\clearpage
\begin{example}
Once again, consider the function 
$$
f(x) = \frac{x^4}{4}+\frac{x^3}{3}-x^2
$$
Use the second derivative test, Theorem~\ref{T:sdt}, to locate the
relative extrema of $f(x)$.  \cite{mooc}
\end{example}

\begin{solution}
Start by computing
\[
f'(x) = x^3 + x^2 -2x \qquad\text{and}\qquad f''(x) = 3x^2 + 2x-2.
\] 
Using the same technique as used in the solution of
Example~\ref{E:relativeextrema}, we find that 
\[
f'(-2) = 0,\qquad f'(0) = 0, \qquad f'(1) = 0. 
\]
Now we'll attempt to use the second derivative test, Theorem~\ref{T:sdt},
\[
f''(-2) = 6, \qquad f''(0) = -2, \qquad f''(1) = 3.
\]
Hence we see that $f(x)$ has a relative minimum at $x=-2$, a relative
maximum at $x=0$, and a relative minimum at $x=1$, see Figure~\ref{figure:SDT(x^4)/4 + (x^3)/3 -x^2}.
\end{solution}
\begin{figure}[H]
    \centering
\begin{tikzpicture}
	\begin{axis}[
            domain=-4:4,
            ymax=7,
            ymin=-4,
            %samples=100,
            axis lines =middle, xlabel=$x$, ylabel=$y$,
            every axis y label/.style={at=(current axis.above origin),anchor=south},
            every axis x label/.style={at=(current axis.right of origin),anchor=west}
          ]
          \addplot [dashed, textColor, smooth] plot coordinates {(-2,-2.667) (-2,6)}; %% {.451};
          \addplot [dashed, textColor, smooth] plot coordinates {(1,0) (1,3)}; %% axis{2.215};

          \addplot [very thick, penColor, smooth] {(x^4)/4 + (x^3)/3 -x^2};
          \addplot [very thick, penColor2, smooth] {3*x^2 + 2*x -2};

          \node at (axis cs:-1.7,-2.7) [anchor=west] {\color{penColor}$f(x)$};  
          \node at (axis cs:-1.5,2) [anchor=west] {\color{penColor2}$f''(x)$};

          \addplot[color=penColor2,fill=penColor2,only marks,mark=*] coordinates{(-2,6)};  %% closed hole
          \addplot[color=penColor2,fill=penColor2,only marks,mark=*] coordinates{(1,3)};  %% closed hole
          \addplot[color=penColor2,fill=penColor2,only marks,mark=*] coordinates{(0,-2)};  %% closed hole
          \addplot[color=penColor,fill=penColor,only marks,mark=*] coordinates{(0,0)};  %% closed hole
          \addplot[color=penColor,fill=penColor,only marks,mark=*] coordinates{(-2,.-2.667)};  %% closed hole
          \addplot[color=penColor,fill=penColor,only marks,mark=*] coordinates{(1,-.4167)};  %% closed hole
        \end{axis}
\end{tikzpicture}
\caption{A plot of $f(x) =x^4/4 + x^3/3 -x^2$ and $f''(x) = 3x^2 + 2x -2$. \cite{mooc}}
\label{figure:SDT(x^4)/4 + (x^3)/3 -x^2}
\end{figure}


\section{Sketching the Plot of a Function} 


In this section, we will give some general guidelines for sketching
the plot of a function.

\begin{proposition}[Procedure for Sketching the Plots of Functions]\hfil
\begin{itemize}
\item Find the $y$-intercept, this is the point $(0,f(0))$. Place this
  point on your graph.
\item Find candidates for vertical asymptotes, these are points where
  $f(x)$ is undefined.
\item Compute $f'(x)$ and $f''(x)$.
\item Find the critical points, the points where $f'(x) = 0$ or
  $f'(x)$ is undefined.
\item Use the second derivative test to identify relative extrema and/or
  find the intervals where your function is increasing/decreasing.
  \clearpage
\item Find the candidates for inflection points, the points where
  $f''(x) = 0$ or $f''(x)$ is undefined.
\item Identify inflection points and concavity.
\item If possible find the $x$-intercepts, the points where $f(x) =
  0$. Place these points on your graph.
\item Find horizontal asymptotes.
\item Determine an interval that shows all relevant behavior.
\end{itemize}
\cite{mooc}
At this point you should be able to sketch the plot of your function.
\end{proposition}

Let's see this procedure in action. We'll sketch the plot of
$2x^3-3x^2-12x$.  Following our guidelines above, we start by
computing $f(0) = 0$.  Hence we see that the $y$-intercept is
$(0,0)$. Place this point on your plot, see Figure~\ref{figure:CS1}. \cite{mooc}
\begin{figure}[H]
    \centering
\begin{tikzpicture}
	\begin{axis}[
            domain=-2:4,
            xmin=-2,
            xmax=4,
            ymax=25,
            ymin=-25,
            axis lines =middle, xlabel=$x$, ylabel=$y$,
            every axis y label/.style={at=(current axis.above origin),anchor=south},
            every axis x label/.style={at=(current axis.right of origin),anchor=west}
          ]
         \addplot[color=penColor,fill=penColor,only marks,mark=*] coordinates{(0,0)};  %% closed hole
        \end{axis}
\end{tikzpicture}
\caption{We start by placing the point $(0,0)$. \cite{mooc}}
\label{figure:CS1}
\end{figure}

Note that there are no vertical asymptotes as our function is defined
for all real numbers.  Now compute $f'(x)$ and $f''(x)$,
\[
f'(x) = 6x^2 -6x -12\qquad\text{and}\qquad f''(x) = 12x-6.
\]

The critical points are where $f'(x) = 0$, thus we need to solve $6x^2
-6x -12 = 0$ for x. Write
\begin{align*}
6x^2 -6x -12 &= 0 \\
x^2 - x -2 &=0\\
(x-2)(x+1) &=0.
\end{align*}
Thus
\[
f'(2) = 0\qquad\text{and}\qquad f'(-1) = 0.
\]
Mark the critical points $x=2$ and $x=-1$ on your plot, see
Figure~\ref{figure:CS2}. \cite{mooc}
\begin{figure}[H]
    \centering
\begin{tikzpicture}
	\begin{axis}[
            domain=-2:4,
            xmin=-2,
            xmax=4,
            ymax=25,
            ymin=-25,
            axis lines =middle, xlabel=$x$, ylabel=$y$,
            every axis y label/.style={at=(current axis.above origin),anchor=south},
            every axis x label/.style={at=(current axis.right of origin),anchor=west}
          ]
         \addplot [dashed, penColor2] plot coordinates {(-1,-25) (-1,25)}; 
         \addplot [dashed, penColor2] plot coordinates {(2,-25) (2,25)}; 
         \addplot[color=penColor,fill=penColor,only marks,mark=*] coordinates{(0,0)};  %% closed hole
        \end{axis}
\end{tikzpicture}
\caption{Now we add the critical points $x=-1$ and $x=2$. \cite{mooc}}
\label{figure:CS2}
\end{figure}

Check the second derivative evaluated at the critical points. In this
case,
\[
f''(-1) = -18 \qquad\text{and}\qquad f''(2) = 18,
\]
hence $x=-1$, corresponding to the point $(-1,7)$ is a relative maximum
and $x=2$, corresponding to the point $(2,-20)$ is relative minimum of
$f(x)$. Moreover, this tells us that our function is increasing on
$[-2,-1)$, decreasing on $(-1,2)$, and increasing on $(2,4]$. Identify
this on your plot, see Figure~\ref{figure:CS3}. \cite{mooc}
\begin{figure}[H]
    \centering
\begin{tikzpicture}
	\begin{axis}[
            axis on top=true,
            domain=-2:4,
            xmin=-2,
            xmax=4,
            ymax=25,
            ymin=-25,
            axis lines =middle, xlabel=$x$, ylabel=$y$,
            every axis y label/.style={at=(current axis.above origin),anchor=south},
            every axis x label/.style={at=(current axis.right of origin),anchor=west}
          ]
          \addplot [->, line width=10, penColor!10!background] plot coordinates {(-2,-25) (-1,7)}; 
          \addplot [->, line width=10, penColor!10!background] plot coordinates {(-1,7) (2,-20)}; 
          \addplot [->, line width=10, penColor!10!background] plot coordinates {(2,-20) (4,25)}; 
          \addplot [dashed, penColor2] plot coordinates {(-1,-25) (-1,25)}; 
          \addplot [dashed, penColor2] plot coordinates {(2,-25) (2,25)}; 
          \addplot [color=penColor,fill=penColor,only marks,mark=*] coordinates{(0,0)};  %% closed hole
          \addplot [color=penColor,fill=penColor,only marks,mark=*] coordinates{(-1,7)};  %% closed hole
          \addplot [color=penColor,fill=penColor,only marks,mark=*] coordinates{(2,-20)};  %% closed hole
          %\addplot [very thick, penColor, samples=100, smooth,domain=(-1.2:-.8)] {2*x^3-3*x^2-12*x};
          %\addplot [very thick, penColor, samples=100, smooth,domain=(1.8:2.2)] {2*x^3-3*x^2-12*x};
        \end{axis}
\end{tikzpicture}
\caption{We have identified the relative extrema of $f(x)$ and where this
  function is increasing and decreasing. \cite{mooc}}
\label{figure:CS3}
\end{figure}


The candidates for the inflection points are where $f''(x) = 0$, thus
we need to solve $12x-6=0$ for $x$.  Write
\begin{align*}
12x-6 &=0\\
x - 1/2 &=0\\
x &=1/2.
\end{align*}
Thus $f''(1/2) = 0$. Checking points, $f''(0) = -6$ and $f''(1) = 6$.
Hence $x=1/2$ is an inflection point, with $f(x)$ concave down to the
left of $x=1/2$ and $f(x)$ concave up to the right of $x=1/2$. We can
add this information to our plot, see Figure~\ref{figure:CS4}. \cite{mooc}

Finally, in this case, $f(x) =2x^3-3x^2-12x$, we can find the
$x$-intercepts. Write
\begin{align*}
2x^3-3x^2-12x &=0\\
x(2x^2 -3x -12) &=0.\\
\end{align*}
Using the quadratic formula, we see that the $x$-intercepts of $f(x)$ are
\[
x = 0, \qquad x= \frac{3-\sqrt{105}}{4}, \qquad x= \frac{3+\sqrt{105}}{4}.
\]
Since all of this behavior as described above occurs on the interval
$[-2,4]$, we now have a complete sketch of $f(x)$ on this interval,
see the figure below. \cite{mooc}
\begin{figure}[H]
    \centering
\begin{tikzpicture}
	\begin{axis}[
            axis on top=true,
            domain=-2:4,
            xmin=-2,
            xmax=4,
            ymax=25,
            ymin=-25,
            axis lines =middle, xlabel=$x$, ylabel=$y$,
            every axis y label/.style={at=(current axis.above origin),anchor=south},
            every axis x label/.style={at=(current axis.right of origin),anchor=west}
          ]
          \addplot [->, line width=10, penColor!10!background] plot coordinates {(-2,-25) (-1,7)}; 
          \addplot [->, line width=10, penColor!10!background] plot coordinates {(-1,7) (2,-20)}; 
          \addplot [->, line width=10, penColor!10!background] plot coordinates {(2,-20) (4,25)}; 
          \addplot [dashed, penColor2] plot coordinates {(-1,-25) (-1,25)}; 
          \addplot [dashed, penColor2] plot coordinates {(2,-25) (2,25)}; 
          \addplot [dashed, penColor2] plot coordinates {(1/2,-25) (1/2,25)}; 
          \addplot [color=penColor,fill=penColor,only marks,mark=*] coordinates{(1/2,-6.5)};  %% closed hole
          \addplot [color=penColor,fill=penColor,only marks,mark=*] coordinates{(0,0)};  %% closed hole
          \addplot [color=penColor,fill=penColor,only marks,mark=*] coordinates{(-1,7)};  %% closed hole
          \addplot [color=penColor,fill=penColor,only marks,mark=*] coordinates{(2,-20)};  %% closed hole
          \addplot [very thick, penColor, samples=100, smooth,domain=(-1.5:3)] {2*x^3-3*x^2-12*x};
        \end{axis}
\end{tikzpicture}
\caption{We identify the inflection point and note that the curve is
  concave down when $x<1/2$ and concave up when $x>1/2$. \cite{mooc}}
\label{figure:CS4}
\end{figure}

\begin{figure}[H]
\centering
\begin{tikzpicture}
	\begin{axis}[
            axis on top=true,
            domain=-2:4,
            xmin=-2,
            xmax=4,
            ymax=25,
            ymin=-25,
            axis lines =middle, xlabel=$x$, ylabel=$y$,
            every axis y label/.style={at=(current axis.above origin),anchor=south},
            every axis x label/.style={at=(current axis.right of origin),anchor=west}
          ]
          \addplot [->, line width=10, penColor!10!background] plot coordinates {(-2,-25) (-1,7)}; 
          \addplot [->, line width=10, penColor!10!background] plot coordinates {(-1,7) (2,-20)}; 
          \addplot [->, line width=10, penColor!10!background] plot coordinates {(2,-20) (4,25)}; 
          \addplot [dashed, penColor2] plot coordinates {(-1,-25) (-1,25)}; 
          \addplot [dashed, penColor2] plot coordinates {(2,-25) (2,25)}; 
          \addplot [dashed, penColor2] plot coordinates {(1/2,-25) (1/2,25)}; 
          \addplot [color=penColor,fill=penColor,only marks,mark=*] coordinates{(1/2,-6.5)};  %% closed hole
          \addplot [color=penColor,fill=penColor,only marks,mark=*] coordinates{(0,0)};  %% closed hole
          \addplot [color=penColor,fill=penColor,only marks,mark=*] coordinates{(-1,7)};  %% closed hole
          \addplot [color=penColor,fill=penColor,only marks,mark=*] coordinates{(2,-20)};  %% closed hole
          \addplot [color=penColor,fill=penColor,only marks,mark=*] coordinates{(-1.812,0)};  %% closed hole
          \addplot [color=penColor,fill=penColor,only marks,mark=*] coordinates{(3.312,0)};  %% closed hole
          \addplot [very thick, penColor, samples=100, smooth,domain=(-2:4)] {2*x^3-3*x^2-12*x};
        \end{axis}
\end{tikzpicture}
\end{figure}
\clearpage

\begin{exercise}
    ~\\\\\-\hspace{0.3cm} \textbf{
        In Exercises 1-4, find all critical points and identify them as relative maximum points, relative minimum points, or neither.
    }\cite{mooc}
    \begin{enumerate} 
        \item $y=x^2-x$
        \item $y=2+3x-x^3$ 
        \item $y=x^3-9x^2+24x$
        \item $f(x) = |x^2 - 121|$
        \item Let $f(x) =a x^2 + bx + c$ with $a\neq 0$. Show that $f(x)$ has exactly one critical point using the first derivative test. Give conditions on $a$ and $b$ which guarantee that the critical point will
        be a maximum. 
    \end{enumerate}
    ~\\\\\-\hspace{0.3cm} \textbf{
        In Exercises 6-9, Sketch the curves via the procedure outlined in this chapter. Clearly identify any interesting features, including relative maximum and minimum points, inflection points, asymptotes, and intercepts.
    }\cite{mooc}
    \begin{enumerate} 
        \setcounter{enumi}{5}
        \item $y= x^5 - x$
        \item $y=2\sqrt{x} - x$
        \item $y=x^5-5x^4+5x^3$
        \item $y = x^2+ 1/x$
    \end{enumerate}
\end{exercise}
