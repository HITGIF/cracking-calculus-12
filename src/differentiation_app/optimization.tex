\chapterimage{img/top.jpg} 
\chapter{Optimization}

\section{Applied Maxima and Mimina Problems}

Many important applied problems involve finding the best way to
accomplish some task. Often this involves finding the maximum or
minimum value of some function: The minimum time to make a certain
journey, the minimum cost for doing a task, the maximum power that can be generated by a device, and so on. These problems can be solved by finding the \B{absolute maxima} and \B{absolute minima} of a function as we had previously discussed. \\

\begin{proposition}[General Strategies for Optimization Problems]\hfil~\\
    \begin{enumerate}
        \item Identify all given quantities and all quantities to be determined. If possible, make a sketch and label it with any relevant measurements.
        \item Write a primary equation for the quantity that is to be maximized or minimized. 
        \item Reduce the primary equation to one having a single independent variable. This may involve the use of secondary equations relating the independent variables of the primary equation.
        \item Determine the feasible domain of the primary equation. That is, determine the values for which the stated problem makes sense.
        \item Determine the desired maximum or minimum value by the calculus techniques.
    \end{enumerate}
\end{proposition}
~\\
Since you are aware of the concept, let's get straight into the examples.
\clearpage
\begin{example}
    Of all rectangles of area $100$ cm$^2$, which has the smallest
    perimeter? \cite{mooc}
    
    \begin{figure}[H]
    \centering
    \begin{tikzpicture}
    \draw [penColor,very thick,fill=fill2] (0,0) rectangle (5,4);
    \node [penColor] at (2.5,2) {$A=100$ cm$^2$};
    \node [right,penColor] at (5,2) {$\frac{100}{x}$ cm};
    \node [below,penColor] at (2.5,0) {$x$ cm};
    \end{tikzpicture}
    \caption{A rectangle with an area of $100$ cm$^2$.}
    \label{fig:rect100}
    \end{figure}
    
    \begin{solution}
    First we draw a picture, see Figure~\ref{fig:rect100}.  If $x$ denotes
    one of the sides of the rectangle, then the adjacent side must be
    $100/x$.
    
    
    The perimeter of this rectangle is given by
    \[
    p(x)=2x+2\frac{100}{x}.
    \]
    We wish to minimize $p(x)$.  Note, not all values of $x$ make sense in
    this problem: lengths of sides of rectangles must be positive, so
    $x>0$. If $x>0$ then so is $100/x$, so we need no second condition on
    $x$.
    
    We next find $p'(x)$ and set it equal to zero. Write
    \[
    p'(x)=2-200/x^2 = 0.
    \]
    Solving for $x$ gives us $x=\pm 10$. We are interested only in $x>0$,
    so only the value $x=10$ is of interest. Since $p'(x)$ is defined
    everywhere on the interval $(0,\infty)$, there are no more critical
    values, and there are no endpoints. Is there a local maximum, minimum,
    or neither at $x=10$? The second derivative is $p''(x)=400/x^3$, and
    $f''(10)>0$, so there is a local minimum. Since there is only one
    critical value, this is also the global minimum, so the rectangle with
    smallest perimeter is the $10$ cm$\times10$ cm square.
    \end{solution}
\end{example}

\begin{example}
    You want to sell a certain number $n$ of items in order to maximize your
    profit.  Market research tells you that if you set the price at \$$1.50$, you
    will be able to sell $5000$ items, and for every $10$ cents you lower the price
    below \$$1.50$ you will be able to sell another $1000$ items.  Suppose that
    your fixed costs (``start-up costs'') total \$$2000$, and the per item cost
    of production (``marginal cost'') is \$$0.50$.  Find the price to set per
    item and the number of items sold in order to maximize profit, and also
    determine the maximum profit you can get. \cite{mooc}\\
    
    \begin{solution}
    The first step is to convert the problem into a function maximization
    problem. The revenue for selling $n$ items at $x$ dollars is given by
    \[
    r(x) = nx
    \]
    and the cost of producing $n$ items is given by
    \[
    c(x) = 2000+0.5 n. 
    \]
    However, from the problem we see that the number of items sold is
    itself a function of $x$,
    \[
    n(x) =5000+1000(1.5-x)/0.10
    \]
    So profit is give by:
    \begin{align*}
    P(x) &= r(x) - c(x)\\
    &= nx - (2000+0.5 n)\\
    &= (5000+1000(1.5-x)/0.10)x - 2000 - 0.5 (5000+1000(1.5-x)/0.10)\\
    &=-10000x^2+25000x-12000. 
    \end{align*}
    We want to know the maximum value of this function when $x$ is between
    0 and $1.5$. The derivative is $P'(x)=-20000x+25000$, which is zero
    when $x=1.25$. Since $P''(x)=-20000<0$, there must be a local maximum
    at $x=1.25$, and since this is the only critical value it must be a
    global maximum as well. Alternately, we could compute $P(0)=-12000$,
    $P(1.25)=3625$, and $P(1.5)=3000$ and note that $P(1.25)$ is the
    maximum of these. Thus the maximum profit is \$$3625$, attained when we
    set the price at \$$1.25$ and sell $7500$ items. 
    \end{solution}
\end{example}

\begin{example} 
    Find the rectangle with largest area that fits inside the graph of the
    parabola $y=x^2$ below the line $y=a$, where $a$ is an unspecified
    constant value, with the top side of the rectangle on the horizontal
    line $y=a$. See Figure~\ref{fig:rectangle parabola}. \cite{mooc}\\
    
    \begin{figure}[H]
    \centering
    \begin{tikzpicture}
        \begin{axis}[
                domain=-3:3, ymin=0, ymax=9, xmin=-3, xmax=3,
                axis lines =center, xlabel=$x$, ylabel=$y$,
                ticks=none,
                every axis y label/.style={at=(current axis.above origin),anchor=south},
                every axis x label/.style={at=(current axis.right of origin),anchor=west},
                axis on top,
              ]
              \addplot [draw=none, fill=fill2, domain=(-1.5:1.5)] {7} \closedcycle;
              \addplot [draw=none, fill=background, domain=(-1.5:1.5)] {2.25} \closedcycle;
    
              \addplot [very thick,penColor2,domain=(-3:3)] {x^2};
              \addplot [very thick,penColor2,domain=(-3:3)] {7};
    
              \addplot [very thick, penColor] plot coordinates {(1.5,2.25) (-1.5,2.25)};
              \addplot [very thick, penColor] plot coordinates {(1.5,2.25) (1.5,7)};
              \addplot [very thick, penColor] plot coordinates {(1.5,7) (-1.5,7)};
              \addplot [very thick, penColor] plot coordinates {(-1.5,7) (-1.5,2.25)};
              
              \addplot [color=penColor,fill=penColor,only marks,mark=*] coordinates{(1.5,2.25)};  %% closed hole          
              \addplot [color=penColor,fill=penColor,only marks,mark=*] coordinates{(-1.5,2.25)};  %% closed hole          
              \addplot [color=penColor,fill=penColor,only marks,mark=*] coordinates{(1.5,7)};  %% closed hole          
              \addplot [color=penColor,fill=penColor,only marks,mark=*] coordinates{(-1.5,7)};  %% closed hole   
    
              \node at (axis cs:0,4.625) [penColor] {$A(x) =$~area};
              \node at (axis cs:1.5,2.25) [anchor=west,penColor] {$(x,x^2)$};
              \node at (axis cs:1.5,7.2) [anchor=west,penColor] {$(x,a)$};
            \end{axis}
    \end{tikzpicture}
    \caption{A plot of the parabola $y=x^2$ along with the line $y=a$ and the rectangle in question.}
    \label{fig:rectangle parabola}
    \end{figure}
    
    \begin{solution}
    We want to maximize value of $A(x)$.  The lower right corner of the
    rectangle is at $(x,x^2)$, and once this is chosen the rectangle is
    completely determined. Then the area is
    \[
    A(x)=(2x)(a-x^2)=-2x^3+2ax.
    \] 
    We want the maximum value of $A(x)$ when $x$ is in $[0,\sqrt{a}]$. You
    might object to allowing $x=0$ or $x=\sqrt{a}$, since then the
    ``rectangle'' has either no width or no height, so is not ``really'' a
    rectangle. But the problem is somewhat easier if we simply allow such
    rectangles, which have zero area as we may then apply the Extreme
    Value Theorem, Theorem~\ref{theorem:evt}.
    
    Setting $0=A'(x)=-6x^2+2a$ we find $x=\sqrt{a/3}$ as the only critical
    point. Testing this and the two endpoints, we have
    $A(0)=A(\sqrt{a})=0$ and $A(\sqrt{a/3})=(4/9)\sqrt{3}a^{3/2}$. Hence,
    the maximum area thus occurs when the rectangle has dimensions
    $2\sqrt{a/3}\times (2/3)a$.
    \end{solution}
\end{example}
\clearpage
\begin{example}
    If you fit the largest possible cone inside a sphere, what fraction of the
    volume of the sphere is occupied by the cone?  (Here by ``cone'' we mean a
    right circular cone, i.e., a cone for which the base is perpendicular to
    the axis of symmetry, and for which the cross-section cut perpendicular to
    the axis of symmetry at any point is a circle.) \cite{mooc}
    
    \begin{figure}[H]
        \centering
    
    \begin{tikzpicture}
    \draw[very thick,penColor2!20!background] (2,0) arc (0:180:2 and .7);% top half of ellipse
    \draw [penColor, very thick] (0,1) ellipse (1.7 and .5);
    \draw[penColor, very thick] (1.7,.95) -- (0,-2);
    \draw[penColor, very thick] (-1.7,.95) -- (0,-2);
    \draw[very thick,penColor2] (-2,0) arc (180:360:2 and .7);% bottom half of ellipse
    
    
    \draw [penColor2, very thick] (0,0) circle [radius=2];
    
    \draw[penColor2, dashed, very thick] (0,0) -- (2,0);
    \draw[penColor, dashed, very thick] (0,1) -- (1.7,1);
    \draw[penColor, dashed, very thick] (0,1) -- (0,-2);
    
    \node [below,penColor2] at (1.5,0) {$R$};
    \node [above,penColor] at (.85,1) {$r$};
    \node [left,penColor] at (0,-.33) {$h$};
    
    \node [penColor,left] at (-1.5,1.42) {$V_{\text{c}} = \frac{\pi r^2h}{3}$};
    \node [penColor2, right] at (1.5,-1.42) {$V_{s} = \frac{4\pi R^3}{3}$};
    \end{tikzpicture}
    \caption{A cone inside a sphere.}
    \label{fig:cone-sphere}
    \end{figure}
    
    \begin{solution}
    Let $R$ be the radius of the sphere, and let $r$ and $h$ be the base
    radius and height of the cone inside the sphere.  Our goal is to
    maximize the volume of the cone: $V_c=\pi r^2h/3$.  The largest $r$
    could be is $R$ and the largest $h$ could be is $2R$.
    
    Notice that the function we want to maximize, $\pi r^2h/3$, depends on
    \textit{two} variables.  Our next step is to find the relationship and
    use it to solve for one of the variables in terms of the other, so as
    to have a function of only one variable to maximize.  In this problem,
    the condition is apparent in the figure, as the upper corner of the
    triangle, whose coordinates are $(r,h-R)$, must be on the circle of
    radius $R$. Write
    \[
    r^2 + (h-R)^2=R^2.
    \] 
    Solving for $r^2$, since $r^2$ is found in the formula for the volume
    of the cone, we find 
    \[
    r^2=R^2-(h-R)^2.
    \]  
    Substitute this into the formula for the volume of the cone to find
    
    \begin{align*}
     V_{\text{c}}(h)&=\pi(R^2-(h-R)^2)h/3 \\
    &=-{\pi\over3}h^3+{2\over3}\pi h^2R
    \end{align*}
    
    We want to maximize $V_{\text{c}}(h)$ when $h$ is between 0 and $2R$.  We
    solve 
    \[
    V_{\text{c}}'(h)=-\pi h^2+(4/3)\pi h R=0,
    \] 
    finding $h=0$ or $h=4R/3$.  We compute
    \[
    V_{\text{c}}(0)=V_{\text{c}}(2R)=0\qquad\text{and}\qquad V_{\text{c}}(4R/3)=(32/81)\pi R^3.
    \] 
    The maximum is the latter. Since the volume of the sphere is $(4/3)\pi
    R^3$, the fraction of the sphere occupied by the cone is
    \[
    \frac{(32/81)\pi R^3}{(4/3)\pi R^3}=\frac{8}{27}\approx 30\%.
    \]
    \end{solution}
\end{example}

    
\begin{example}
    You are making cylindrical containers to contain a given volume.  Suppose
    that the top and bottom are made of a material that is $N$ times as
    expensive (cost per unit area) as the material used for the lateral side of
    the cylinder.  Find (in terms of $N$) the ratio of height to base radius of
    the cylinder that minimizes the cost of making the containers. \cite{mooc}
    
    \begin{figure}[H]
        \centering
    \begin{tikzpicture}
    \draw[penColor,very thick] (0,2) ellipse (2 and .7);
    \draw[very thick,penColor!20!background] (2,-2) arc (0:180:2 and .7);% top half of ellipse
    \draw[very thick,penColor] (-2,-2) arc (180:360:2 and .7);% bottom half of ellipse
    
    \draw[penColor, very thick] (2,2) -- (2,-2);
    \draw[penColor, very thick] (-2,2) -- (-2,-2);
    
    \draw[penColor, dashed, very thick] (0,2) -- (2,2);
    \draw[penColor, dashed, very thick] (0,2) -- (0,-2);
    
    \node [above,penColor] at (1,2) {$r$};
    \node [left,penColor] at (0,-.33) {$h$};
    \node [penColor,right] at (2,-1.42) {$V = \pi r^2h$};
    \end{tikzpicture}
    \caption{A cylinder with radius $r$, height $h$, volume $V$, $c$ for
      the cost per unit area of the lateral side of the cylinder.}
    \label{fig:cylinder}
    \end{figure}
    
    \begin{solution}
    First we draw a picture, see Figure~\ref{fig:cylinder}.  Now we can
    write an expression for the cost of materials:
    \[
      C = 2\pi crh+2\pi r^2Nc.
    \]
    Since we know that $V=\pi r^2h$, we can use this relationship to
    eliminate $h$ (we could eliminate $r$, but it's a little easier if we
    eliminate $h$, which appears in only one place in the above formula
    for cost).  We find
    \begin{align*}
    C(r)&=2c\pi r\frac{V}{\pi r^2}+2Nc\pi r^2\\
    &=\frac{2cV}{r}+2Nc\pi r^2.
    \end{align*}
    We want to know the minimum value of this function when $r$ is in
    $(0,\infty)$. Setting
    \[
    C'(r)=-2cV/r^2+4Nc\pi r =0
    \]
    we find $r=\sqrt[3]{V/(2N\pi)}$.  Since $C''(r)=4cV/r^3+4Nc\pi$ is
      positive when $r$ is positive, there is a local minimum at the
      critical value, and hence a global minimum since there is only one
      critical value.
    
    Finally, since $h=V/(\pi r^2)$, 
    \begin{align*}
    \frac{h}{r}&=\frac{V}{\pi r^3}\\ 
    &=\frac{V}{\pi(V/(2N\pi))}\\ 
    &=2N,
    \end{align*}
    so the minimum cost occurs when the height $h$ is $2N$ times the
    radius. If, for example, there is no difference in the cost of
    materials, the height is twice the radius.
    \end{solution}
\end{example}
    
\begin{example}\label{exam:sand and road} Suppose you want to reach a point $A$ that is located across the
    sand from a nearby road, see Figure~\ref{fig:roadsand}.  Suppose that
    the road is straight, and $b$ is the distance from $A$ to the closest
    point $C$ on the road.  Let $v$ be your speed on the road, and let
    $w$, which is less than $v$, be your speed on the sand.  Right now you
    are at the point $D$, which is a distance $a$ from $C$.  At what point
    $B$ should you turn off the road and head across the sand in order to
    minimize your travel time to $A$? \cite{mooc}
    
    \begin{figure}[H]
        \centering
    
    \begin{tikzpicture}
    \draw[fill2, fill=fill2] (0,0) rectangle (6,4);
    \draw[fill1, fill=fill1] (0,.4) rectangle (6,1);
    
    \node[penColor] at (.5,.75) {\scalebox{-2}[2] \Bicycle};
    
    \draw [penColor, fill] (1,.75) circle [radius=.07];
    \draw [penColor, fill] (2.5,.75) circle [radius=.07];
    \draw [penColor, fill] (5,.75) circle [radius=.07];
    \draw [penColor, fill] (5,3) circle [radius=.07];
    
    \draw[black, very thick, ->] (1.2,.75) -- (2.3,.75);
    \draw[black, very thick, ->] (2.6,.85) -- (4.9,2.9);
    
    \draw[penColor, very thick, dashed] (5,.75) -- (5,3);
    \draw[penColor, very thick, dashed] (5,.75) -- (2.5,.75);
    \draw[penColor, very thick, dashed] (1,.4) -- (5,.4);
    
    \node [right,penColor] at (5,3) {$A$};
    \node [below,penColor] at (1,.75) {$D$};
    \node [below,penColor] at (2.5,.75) {$B$};
    \node [below,penColor] at (5,.75) {$C$};
    \node [right,penColor] at (5,2) {$b$};
    \node [above,penColor] at (4,.75) {$x$};
    \node [left,black] at (3.8,2) {$w$};
    \node [above,black] at (1.75,.75) {$v$};
    \node [below,penColor] at (3,.4) {$a$};
    \end{tikzpicture}
    \caption{A road where one travels at rate $v$, with sand where one
      travels at rate $w$. Where should one turn off of the road to
      minimize total travel time from $D$ to $A$?}
    \label{fig:roadsand}
    \end{figure}
    
    \begin{solution}
    Let $x$ be the distance short of $C$ where you turn off, the distance
    from $B$ to $C$.  We want to minimize the total travel time.  Recall
    that when traveling at constant velocity, time is distance divided by
    velocity.
    
    You travel the distance from $D$ to $B$ at speed $v$, and then the
    distance from $B$ to $A$ at speed $w$.  The distance from $D$ to $B$
    is $a-x$. By the Pythagorean theorem, the distance from $B$ to $A$
    is
    \[
    \sqrt{x^2+b^2}.
    \] 
    Hence the total time for the trip is
    \[
       T(x)=\frac{a-x}{v}+\frac{\sqrt{x^2+b^2}}{w}.
    \]
    We want to find the minimum value of $T$ when $x$ is between 0 and
    $a$.  As usual we set $T'(x)=0$ and solve for $x$. Write
    \[
      T'(x)=-\frac{1}{v}+\frac{x}{w\sqrt{x^2+b^2}} =0.
    \]
    We find that 
    \[
    x=\frac{wb}{\sqrt{v^2-w^2}}
    \]
    Notice that $a$ does not appear in the last expression, but $a$ is not
    irrelevant, since we are interested only in critical values that are
    in $[0,a]$, and $wb/\sqrt{v^2-w^2}$ is either in this interval or not.
    If it is, we can use the second derivative to test it:
    \[
    T''(x) = {b^2\over (x^2+b^2)^{3/2}w}.
    \]
    Since this is always positive there is a local minimum at the critical
    point, and so it is a global minimum as well.
    
    If the critical value is not in $[0,a]$ it is larger than $a$. In this
    case the minimum must occur at one of the endpoints. We can compute
    \begin{align*}
    T(0)&={a\over v}+{b\over w} \\
    T(a)&={\sqrt{a^2+b^2}\over w} 
    \end{align*}
    but it is difficult to determine which of these is smaller by direct
    comparison. If, as is likely in practice, we know the values of $v$,
    $w$, $a$, and $b$, then it is easy to determine this. With a little
    cleverness, however, we can determine the minimum in general. We have seen that
    $T''(x)$ is always positive, so the derivative $T'(x)$ is always increasing.
    We know that at $wb/\sqrt{v^2-w^2}$ the derivative is zero, so for
    values of $x$ less than that critical value, the derivative is
    negative. This means that $T(0)>T(a)$, so the minimum occurs when $x=a$.
    
    So the upshot is this: If you start farther away from $C$ than
    $wb/\sqrt{v^2-w^2}$ then you always want to cut across the sand 
    when you are a distance $wb/\sqrt{v^2-w^2}$ from point $C$. If you
    start closer than this to $C$, you should cut directly across the sand.
    \end{solution}
\end{example}
    

\begin{exercise}~\\
    \begin{enumerate} 
        \item Find the dimensions of the rectangle of largest area having fixed perimeter $100$. \cite{mooc}
        \item A box with square base and no top is to hold a volume $100$.  Find the dimensions of the box that requires the least material for the five sides.  Also find the ratio of height to side of the base. \cite{mooc}
        \item A box with square base and no top is to hold a volume $V$.  Find (in terms of $V$) the dimensions of the box that requires the least material for the five sides.  Also find the ratio of height to side of the base.  (This ratio will not involve $V$.) \cite{mooc}
        \item Marketing tells you that if you set the price of an item at \$10 then you will be unable to sell it, but that you can sell 500 items for each dollar below \$10 that you set the price.  Suppose your fixed costs total \$3000, and your marginal cost is \$2 per item.  What is the most profit you can make? \cite{mooc}
        \item In Example~\ref{exam:sand and road}, what happens if $w\ge v$? \cite{mooc}
        \item For a cylinder with surface area $50$, including the top and the bottom, find the ratio of height to base radius that maximizes the volume. \cite{mooc}
        \item Given a right circular cone, you put an upside-down cone inside it so that its vertex is at the center of the base of the larger cone and its base is parallel to the base of the larger cone.  If you choose the upside-down cone to have the largest possible volume, what fraction of the volume of the larger cone does it occupy?  (Let $H$ and $R$ be the height and base radius of the larger cone, and let $h$ and $r$ be the height and base radius of the smaller cone. Hint: Use similar triangles to get an equation relating $h$ and $r$.) \cite{mooc}
        \item Two electrical charges, one a positive charge $A$ of magnitude $a$ and the other a negative charge $B$ of magnitude $b$, are located a distance $c$ apart.  A positively charged particle $P$ is situated on the line between $A$ and $B$.  Find where $P$ should be put so that the pull away from $A$ towards $B$ is minimal.  Here assume that the force from each charge is proportional to the strength of the source and inversely proportional to the square of the distance from the source. \cite{mooc}
        \item If you fit the cone with the largest possible surface area (lateral area plus area of base) into a sphere, what percent of the volume of the sphere is occupied by the cone? \cite{mooc}
    \end{enumerate}
\end{exercise}
\clearpage
