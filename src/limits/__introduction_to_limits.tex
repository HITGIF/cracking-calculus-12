\chapterimage{img/lim_intro.jpg} 
\chapter{Introduction to Limits}

\section{The Basic Idea of Limits}

Consider the function:
$$f(x)=\protect\frac{x^2 - 3x + 2}{x-2}$$

\begin{multicols}{2}
\begin{figure}[H]
	\begin{tikzpicture}
		\begin{axis}[
				domain=-2:4,
				axis lines =middle, xlabel=$x$, ylabel=$y$,
				every axis y label/.style={at=(current axis.above origin),anchor=south},
				every axis x label/.style={at=(current axis.right of origin),anchor=west},
				grid=both,
				grid style={dashed, gridColor},
				xtick={-2,...,4},
				ytick={-3,...,3},
			  ]
		  		\addplot [very thick, penColor, smooth] {x-1};
			  	\addplot[color=penColor,fill=background,only marks,mark=*] coordinates{(2,1)};  %% open hole
		\end{axis}
	\end{tikzpicture}
	\caption{A plot of $f(x)=\protect\frac{x^2 - 3x + 2}{x-2}$.}
	\label{plot:(x^2 - 3x + 2)/(x-2)}
\end{figure}

\vspace*{\fill}
\begin{table}[H]
	\centering
	\begin{tabular}{l l}
		\toprule
		\textbf{$x$} & \textbf{$f(x)$} \\
		\midrule
 		1.7 	&  0.7 		\\
 		1.9 	&  0.9 		\\
 		1.99 	&  0.99 	\\
 		1.999 	&  0.999 	\\
		2 		&  \text{undefined} \\
		2.001	&  1.001	\\
		2.01	&  1.01		\\
		2.1 	&  1.1 		\\
		2.3 	&  1.3 		\\
		\bottomrule
	\end{tabular}
\caption{Values of $f(x)=\protect\frac{x^2 - 3x + 2}{x-2}$.}
\end{table}
\vspace*{\fill}
\end{multicols}

While $f(x)$ is undefined at $x = 2$, we can still plot $f(x)$ at other 
values, see Figure 1.1. Examining Table 1.1, we see that as $x$ 
approaches $2$, $f(x)$ approaches $1$. We write this:
$$\text{As  } x \to 2\text{,  }f(x) \to 1
\qquad\text{or}\qquad
\lim_{x\to 2} f(x) = 1$$

Intuitively, $\lim\limits_{x\to a} f(x) = L$ when the value of $f(x)$ can be 
made arbitrarily close to $L$ by making $x$ sufficiently close, but not 
equal to, $a$. This leads us to the formal definition of a limit.


\begin{definition}[Limit]
	The \textbf{limit} of $f(x)$ as $x$ approaches $a$ is $L$,
	$$\lim_{x\to a} f(x) = L,$$
	if for every $\epsilon > 0$ there is a $\delta > 0$ such that 
	whenever
	$$0 < |x-a| < \delta,
	\qquad\text{we have}\qquad
	|f(x) - L| < \epsilon$$
	If not such value of $L$ can be found, the the $\lim\limits_{x\to a} f(x)$
	\textbf{does not exist}.
\end{definition}

The geometric interpretation of this definition can be seen in 
Figure 1.2.

\begin{figure}[H]
	\centering
	\begin{tikzpicture}
		\begin{axis}[
            domain=0:2, 
            axis lines =left, xlabel=$x$, ylabel=$y$,
            every axis y label/.style={at=(current axis.above origin),anchor=south},
            every axis x label/.style={at=(current axis.right of origin),anchor=west},
            xtick={0.7,1,1.3}, ytick={3,4,5},
            xticklabels={$a-\delta$,$a$,$a+\delta$}, yticklabels={$L-\epsilon$,$L$,$L+\epsilon$},
            axis on top,
          ]          
			\addplot [color=textColor, fill=fill2, smooth, domain=(0:1.570)] {5} \closedcycle;
			\addplot [color=textColor, dashed, fill=fill1, smooth, domain=(0:1.3)] {4.537} \closedcycle;
			\addplot [color=textColor, dashed, fill=fill2, domain=(0:.7)] {3.283} \closedcycle;       
			\addplot [textColor, very thick, smooth, domain=(0:1)] {4};
			\addplot [color=textColor, fill=background, smooth, domain=(0:0.607)] {3} \closedcycle;
			\addplot [draw=none, fill=background, smooth] {x*(x-2)^2+3*x} \closedcycle;
			\addplot [fill=fill1, draw=none, domain=.7:1.3] {x*(x-2)^2+3*x} \closedcycle;
			\addplot [textColor, very thick] plot coordinates {(1,0) (1,4)};
			\addplot [textColor] plot coordinates {(.7,0) (.7,3.283)};
			\addplot [textColor] plot coordinates {(1.3,0) (1.3,4.537)};
	  		\addplot [very thick,penColor, smooth] {x*(x-2)^2+3*x};
        \end{axis}
	\end{tikzpicture}
	\caption{A geometric interpretation of the
	  $(\epsilon,\delta)$-criterion for limits.  If $0<|x-a|<\delta$, then we have that $a
	  -\delta < x < a+\delta$. In our diagram, we see that for all such
	  $x$ we are sure to have $L - \epsilon< f(x) < L+\epsilon$, and hence
	  $|f(x) - L|<\epsilon$.}
	\label{figure:epsilon-delta}
\end{figure}
~\newline
\begin{example}
	Let $f(x) = \lfloor x\rfloor$. Explain why the limit
	$$\lim_{x\to 2} f(x)$$
	\quad does not exist.
	~\newline
	\begin{solution}~\newline
		The function $\lfloor x \rfloor$ is the function that returns the
		greatest integer less than or equal to $x$. Since $f(x)$ is defined
		for all real numbers, one might be tempted to think that the limit
		above is simply $f(2) = 2$. However, this is not the case.  If $x<2$,
		then $f(x) =1$. Hence if $\epsilon = .5$, we can \textbf{always} find
		a value for $x$ (just to the left of $2$) such that
		$$
		0< |x -2|< \delta, \qquad\text{where} \qquad \epsilon < |f(x)-2|.
		$$
		On the other hand, $\lim\limits_{x\to 2} f(x)\ne 1$, as in this case if
		$\epsilon=.5$, we can \textbf{always} find a value for $x$ (just to
		the right of $2$) such that
		$$
		0<|x- 2|<\delta, \qquad\text{where} \qquad  \epsilon<|f(x)-1|.
		$$
		In fact, no matter what value one
		chooses for $\lim\limits_{x\to 2} f(x)$, we will always have a similar
		issue.
	\end{solution}
\end{example}

Limits may not exist even if the formula for the function looks normal.
~\newline
\begin{example}
	Let $f(x) = \sin\left(\frac{1}{x}\right)$. Explain why the limit
	$$\lim_{x\to 0} f(x)$$
	\quad does not exist.
	~\newline
	\begin{solution}~\newline
		In this case $f(x)$ oscillates ``wildly'' as $x$ approaches $0$, see
		Figure~\ref{plot:sin 1/x}. In fact, one can show that for any given
		$\delta$, There is a value for $x$ in the interval
		$$
		0-\delta < x < 0+\delta
		$$
		such that $f(x)$ is \textbf{any} value in the interval $[-1,1]$. Hence
		the limit does not exist.
	\end{solution}
	\begin{figure}[H]
		\centering
		\begin{tikzpicture}
			\begin{axis}[
					domain=-.2:.2,    
					samples=500,
					axis lines =middle, xlabel=$x$, ylabel=$y$,
					yticklabels = {}, 
					every axis y label/.style={at=(current axis.above origin),anchor=south},
					every axis x label/.style={at=(current axis.right of origin),anchor=west},
					clip=false,
				  ]
			  \addplot [very thick, penColor, smooth, domain=(-.2:-.02)] {sin(deg(1/x))};
				  \addplot [very thick, penColor, smooth, domain=(.02:.2)] {sin(deg(1/x))};
			  \addplot [color=penColor, fill=penColor, very thick, smooth,domain=(-.02:.02)] {1} \closedcycle;
				  \addplot [color=penColor, fill=penColor, very thick, smooth,domain=(-.02:.02)] {-1} \closedcycle;
				\end{axis}
		\end{tikzpicture}
		\caption{A plot of $f(x)=\protect\sin\left(\frac{1}{x}\right)$.}
		\label{plot:sin 1/x}
		\end{figure}
\end{example}

Sometimes the limit of a function exists from one side or the other
(or both) even though the limit does not exist. Since it is useful to
be able to talk about this situation, we introduce the concept of a
\textit{one-sided limit}\index{one-sided limit}:

\begin{definition}[One-Sided Limit]
	We say that the \textbf{limit} of $f(x)$ as $x$ 
	goes to $a$ from the \textbf{left} is $L$,
	$$\lim_{x\to a^-}f(x)=L$$
	if for every $\epsilon>0$ there is a $\delta > 0$ so that 
	whenever $x< a$ and 
	$$a-\delta < x \qquad\text{we have}\qquad |f(x)-L|<\epsilon.$$

	We say that the \textbf{limit} of $f(x)$ as $x$ goes to $a$ 
	from the \textbf{right} is $L$,
	$$\lim_{x\to a^+}f(x)=L$$
	if for every $\epsilon>0$ there is a $\delta > 0$ so that 
	whenever $x > a$ and 
	$$x<a+\delta \qquad\text{we have}\qquad |f(x)-L|<\epsilon.$$
\end{definition}

\begin{exercise}
    ~\\

    \begin{enumerate} 
		% \item Use a table and a calculator to estimate $\lim\limits_{x\to 0} \frac{\sin(x)}{x}$.
        % \item Use a table and a calculator to estimate $\lim\limits_{x\to 0} \frac{\sin(2x)}{x}$.
        % \item Use a table and a calculator to estimate $\lim\limits_{x\to 0^+} \frac{x}{\sin\left(\frac{x}{3}\right)}$.
        % \item Use a table and a calculator to estimate $\lim\limits_{x\to 0^-} \frac{\tan(3x)}{\tan(5x)}$.
		% \item Use a table and a calculator to estimate $\lim\limits_{x\to 0^-} \frac{2^x-1}{x}$.
		\item Use the definition of limits to explain why $\lim\limits_{x\to 0 } x\sin
		\left( {1\over x}\right) = 0$.  Hint: Use the fact that $|\sin(a) |\le 1$ for any real number $a$.
		\item Use the definition of limits to explain why $\lim\limits_{x\to 4^+} (2x-5) = 3$.
		\item Use the definition of limits to explain why $\lim\limits_{x\to -2} \pi = \pi$.
		\item Use the definition of limits to explain why $\lim\limits_{x\to 4} x^3 = 64$.
		\item Use the definition of limits to explain why $\lim\limits_{x\to 9^-} \frac{x-9}{\sqrt{x}-3} = 6$.
        \item Sketch a plot of $f(x) = \dfrac{x}{|x|}$ and explain why $\lim\limits_{x\to 0} \frac{x}{|x|}$ does not exist.
    \end{enumerate}

\end{exercise}
