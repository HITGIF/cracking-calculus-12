\chapterimage{img/flow.jpg} 
\chapter{Continuity}
\section{Continuity at a Point and on an Interval}

Informally, a function is continuous if you can “draw it” without “lifting your pencil.” We need a formal definition. \cite{mooc}

\begin{definition}[Continuity at a Point]
    In order for a functiuon $f(x)$ to be continuous at a point $x=c$, it must fulfill \I{all three} of the following conditions:
    \begin{enumerate}
        \item $f(x)$ exists.
        \item $\Lim{x\to c}f(x)$ exists.
        \item $\Lim{x\to c}f(x) =f(x)$.
    \end{enumerate}
    \cite{ap}
\end{definition}

\begin{example}
    Find the discontinuities (the values for $x$ where a function is not
continuous) for the function given in Figure~\ref{plot:discontinuous-function}. \cite{mooc}
    \begin{figure}[H]
        \centering
        \begin{tikzpicture}
            \begin{axis}[
                    domain=0:10,
                    ymax=5,
                    ymin=0,
                    samples=100,
                    axis lines =middle, xlabel=$x$, ylabel=$y$,
                    height=5cm,
                    every axis y label/.style={at=(current axis.above origin),anchor=south},
                    every axis x label/.style={at=(current axis.right of origin),anchor=west}
                  ]
              \addplot [very thick, penColor, smooth, domain=(4:10)] {3 + sin(deg(x*2))/(x-1)};
                  \addplot [very thick, penColor, smooth, domain=(0:4)] {1};
                  \addplot[color=penColor,fill=background,only marks,mark=*] coordinates{(4,3.30)};  %% open hole
                  \addplot[color=penColor,fill=background,only marks,mark=*] coordinates{(6,2.893)};  %% open hole
                  \addplot[color=penColor,fill=penColor,only marks,mark=*] coordinates{(4,1)};  %% closed hole
                  \addplot[color=penColor,fill=penColor,only marks,mark=*] coordinates{(6,2)};  %% closed hole
                \end{axis}
        \end{tikzpicture}
        \caption{A plot of a function with discontinuities at $x=4$ and $x=6$. \cite{mooc}}
        \label{plot:discontinuous-function}
    \end{figure}
    \begin{solution}
        From Figure~\ref{plot:discontinuous-function} we see that $\Lim{x\to 4} f(x)$ does not exist as
        \[
        \Lim{x\to 4^-}f(x) = 1\qquad\text{and}\qquad \Lim{x\to 4^+}f(x) \approx 3.5
        \]
        Hence $\Lim{x\to 4} f(x) \ne f(4)$, and so $f(x)$ is not
        continuous at $x=4$.
        
        We also see that $\Lim{x\to 6} f(x) \approx 3$ while $f(6) =
        2$. Hence $\Lim{x\to 6} f(x) \ne f(6)$, and so $f(x)$ is not
        continuous at $x=6$.
    \end{solution}
\end{example}

Building from the definition of \textit{continuous at a point}, we can now define what it means for a function to be \textit{continuous} on an interval. \cite{mooc}

\begin{definition} [Continuity at an Open Interval]
    A function $f$ is \textbf{continuous on  open interval $(a,b)$} if it is continuous at every point in the interval. A function that is continuous on the entire real line $(-\infty,\infty)$ is \B{everywhere continuous}.
    \\\cite{mooc}
\end{definition}

\begin{definition} [Continuity at Closed Interval]
    A function $f$ is \textbf{continuous on an closed interval $[a,b]$} if it is continuous on the open interval $(a, b)$, and
    $$\Lim{x\to a^+}f(x)=f(a)\qquad\text{and}\qquad\Lim{x\to b^-}f(x)=f(b)$$
    The function $f$ is \B{continuous from the right} at $a$ and \B{continuous from the left} at $b$.
    \\\cite{ci}
\end{definition}

In particular, we should note that if a function is not defined on an interval, then it \textbf{cannot} be continuous on that interval. \cite{mooc}

\begin{example}
Consider the function
\[
f(x) = 
\begin{cases}
\sqrt[5]{x}\sin\left(\frac{1}{x}\right) & \text{if $x \ne 0$,}\\
0 & \text{if $x = 0$,}
\end{cases}
\]
see Figure~\ref{plot:sqrt[5]xsin 1/x}. Is this function continuous? \cite{mooc}

\begin{figure}[H]
    \centering
    \begin{tikzpicture}
        \begin{axis}[
                domain=-.2:.2,    
                samples=500,
                height=5cm,
                axis lines =middle, xlabel=$x$, ylabel=$y$,
                yticklabels = {}, 
                every axis y label/.style={at=(current axis.above origin),anchor=south},
                every axis x label/.style={at=(current axis.right of origin),anchor=west},
                clip=false,
            ]
        \addplot [very thick, penColor, smooth, domain=(-.2:-.02)] {abs(x)^(1/5)*sin(deg(1/x))};
        \addplot [very thick, penColor, smooth, domain=(.02:.2)] {x^(1/5)*sin(deg(1/x))};
    \addplot [color=penColor, fill=penColor, very thick, smooth,domain=(-.02:.02)] {abs(x)^(1/5)} \closedcycle;
        \addplot [color=penColor, fill=penColor, very thick, smooth,domain=(-.02:.02)] {-abs(x)^(1/5)} \closedcycle;
            \end{axis}
    \end{tikzpicture}
    \caption[A continuous function.]{A plot of
$
f(x)=
\begin{cases}
\sqrt[5]{x}\sin\left(\frac{1}{x}\right) & \text{if $x \ne 0$,}\\
 0 & \text{if $x = 0$.}
\end{cases}
$
}
    \label{plot:sqrt[5]xsin 1/x}
\end{figure}

    \begin{solution}
    Considering $f(x)$, the only issue is when $x=0$. We must show that
    $\Lim{x\to 0} f(x) = 0$. Note
    \[
    -|\sqrt[5]{x}|\le f(x) \le |\sqrt[5]{x}|.
    \]
    Since
    \[
    \Lim{x\to 0} -|\sqrt[5]{x}| = 0 = \Lim{x\to 0}|\sqrt[5]{x}|,
    \]
    we see by the Squeeze Theorem, that
    $\Lim{x\to 0} f(x) = 0$. Hence $f(x)$ is continuous.

    Here we see how the informal definition of continuity being that you
    can ``draw it'' without ``lifting your pencil'' differs from the
    formal definition.
    \end{solution}

\end{example}

\section{Types of Discontinuity}
There are three typoes of discontinuity you have to know: jump, essential, removable.

\begin{definition}[Jump Discontinuity]
    A \B{jump} discontinuity occurs when the curve "breaks" at a particular place and starts somewhere else. The limits from the left and the right both exist, but they will not match.
    \\\cite{ap}
\end{definition}

\begin{multicols}{2}
\begin{figure}[H]
    \centering
    \begin{tikzpicture}
        \begin{axis}[
                domain=0:10,
                ymax=5,
                ymin=0,
                xticklabels={,,},
                yticklabels={,,},
                samples=100,
                axis lines =middle, xlabel=$x$, ylabel=$y$,
                height=5cm,
                every axis y label/.style={at=(current axis.above origin),anchor=south},
                every axis x label/.style={at=(current axis.right of origin),anchor=west}
              ]
          \addplot [very thick, penColor, smooth, domain=(4:10)] {3 + sin(deg(x*2))/(x-1)};
              \addplot [very thick, penColor, smooth, domain=(0:4)] {1};
              \addplot[color=penColor,fill=background,only marks,mark=*] coordinates{(4,3.30)};  %% open hole
              \addplot[color=penColor,fill=penColor,only marks,mark=*] coordinates{(4,1)};  %% closed hole
            \end{axis}
    \end{tikzpicture}
    \caption{Jump Discontinuity \cite{mooc}}
    \label{plot:jump}
\end{figure}
\begin{figure}[H]
    \centering
    \begin{tikzpicture}
        \begin{axis}[
                domain=1:4,
                ymax=20,
                ymin=-10,
                xticklabels={,,},
                yticklabels={,,},
                height=5cm,
                samples=100,
                axis lines =middle, xlabel=$x$, ylabel=$y$,
                every axis y label/.style={at=(current axis.above origin),anchor=south},
                every axis x label/.style={at=(current axis.right of origin),anchor=west}
              ]
          \addplot [very thick, penColor, smooth, domain=(0:.9)] {(6*x-9)/(x-1)};
              \addplot [very thick, penColor, smooth, domain=(1.1:3)] {(6*x-9)/(x-1)};
              \addplot [textColor, dashed] plot coordinates {(1,-10) (1,20)};
            \end{axis}
    \end{tikzpicture}
    \caption{Essential Discontinuity \cite{mooc}}
    \label{plot:essential}
\end{figure}
\end{multicols}
\begin{definition}[Essential Discontinuity]
    An \B{essential} discontinuity occurs when the curve has a vertical asymptote.
    \label{essential discontinuity}
    \\\cite{ap}
\end{definition}
\begin{figure}[H]
    \centering
    \begin{tikzpicture}
        \begin{axis}[
                domain=0:10,
                ymax=5,
                ymin=0,
                samples=100,
                xticklabels={,,},
                yticklabels={,,},
                axis lines =middle, xlabel=$x$, ylabel=$y$,
                height=5cm,
                every axis y label/.style={at=(current axis.above origin),anchor=south},
                every axis x label/.style={at=(current axis.right of origin),anchor=west}
              ]
          \addplot [very thick, penColor, smooth, domain=(0:10)] {3 + sin(deg(x*2))/(x+1)};
              \addplot[color=penColor,fill=background,only marks,mark=*] coordinates{(6,2.893)};  %% open hole
              \addplot[color=penColor,fill=penColor,only marks,mark=*] coordinates{(6,2)};  %% closed hole
            \end{axis}
    \end{tikzpicture}
    \caption{Removable Discontinuity \cite{mooc}}
    \label{plot:removable}
\end{figure}
\begin{definition}[Removable Discontinuity]
    An \B{removable} discontinuity occurs when the curve has a "hole" in it. It is "removable" because one can remove the discontinuity by properly defining the value (filling the hole).
    \\\cite{ap}
\end{definition}

\section{The Intermediate Value Theorem}
We close with a useful theorem about continuous functions:

\begin{theorem}[Intermediate Value Theorem]\label{theorem:IVT}~\\
If $f(x)$ is a continuous function for all $x$ in the closed interval $[a,b]$ and $d$ is between $f(a)$ and $f(b)$, then there is a number $c$ in $[a, b]$ such that $f(c) = d$.
\\\cite{mooc}
\end{theorem}

In Figure~\ref{figure:intermediate-value}, we see a geometric
interpretation of this theorem.

\begin{figure}[H]
    \centering
    \begin{tikzpicture}
        \begin{axis}[
                domain=0:6, ymin=0, ymax=2.2,xmax=6,
                axis lines =left, xlabel=$x$, ylabel=$y$,
                every axis y label/.style={at=(current axis.above origin),anchor=south},
                every axis x label/.style={at=(current axis.right of origin),anchor=west},
                xtick={1,3.597,5}, ytick={.203,1,1.679},
                xticklabels={$a$,$c$,$b$}, yticklabels={$f(a)$,$f(c)=d$,$f(b)$},
                axis on top,
            ]
            \addplot [draw=none, fill=fill2, domain=(0:7)] {1.679} \closedcycle;
            \addplot [draw=none, fill=background, domain=(0:7)] {.203} \closedcycle;
            \addplot [textColor,dashed] plot coordinates {(0,1.679) (6,1.679)};
            \addplot [textColor,dashed] plot coordinates {(0,.203) (6,.203)};
            \addplot [textColor,dashed] plot coordinates {(5,0) (5,1.679)};
            \addplot [textColor,dashed] plot coordinates {(1,0) (1,.203)};
            \addplot [textColor,dashed] plot coordinates {(3.587,0) (3.597,1)};
            \addplot [penColor2,domain=(0:6)] {1};
            \addplot [very thick,penColor, smooth,domain=(0:2.5)] {sin(deg((x - 4)/2)) + 1.2};
            \addplot [very thick,penColor, smooth,domain=(4:6)] {sin(deg((x - 4)/2)) + 1.2};
            \addplot [very thick,dashed,penColor!50!background, smooth,domain=(2.5:4)] {sin(deg((x - 4)/2)) + 1.2}; 
            \addplot [color=penColor!50!background,fill=penColor!50!background,only marks,mark=*] coordinates{(3.587,1)};  %% closed hole          
            \addplot [color=penColor,fill=penColor,only marks,mark=*] coordinates{(1,.203)};  %% closed hole          
            \addplot [color=penColor,fill=penColor,only marks,mark=*] coordinates{(5,1.679)};  %% closed hole          
            \end{axis}
    \end{tikzpicture}
    \caption{A geometric interpretation of the Intermediate Value
    Theorem. The function $f(x)$ is continuous on the interval
    $[a,b]$. Since $d$ is in the interval $[f(a),f(b)]$, there exists a
    value $c$ in $[a,b]$ such that $f(c) = d$. \cite{mooc}}
    \label{figure:intermediate-value}
\end{figure}

\begin{example} 
    Explain why the function $f(x) =x^3 + 3x^2+x-2$ has a root between 0 and 1. \cite{mooc}\\
    \begin{solution}~\\
        By Theorem~\ref{theorem:limit-laws}, $\Lim{x\to a} f(x) = f(a)$, for all real values of $a$, and hence $f$ is continuous.  Since $f(0)=-2$ and $f(1)=3$, and $0$ is between $-2$ and $3$, by the Intermediate Value Theorem, Theorem~\ref{theorem:IVT}, there is a $c\in[0,1]$ such
        that $f(c)=0$.
    \end{solution}
\end{example}
    
\begin{exercise}
    ~\\\-\hspace{0.3cm} \textbf{
        In Exercises 1–4, find the x-values (if any) at which f is not continuous. Which of the discontinuities are removable?
    }\cite{ci}\\
    \begin{enumerate} 
		\item $x^2-2x+1$
		\item $\dfrac{x}{x^2-x}$
		\item $\dfrac{x+2}{x^2-3x-10}$
		\item $\dfrac{|x+7|}{x+7}$
    \end{enumerate}
    ~\\\-\hspace{0.3cm} \textbf{
        In Exercises 5–6, find the constant $a$, or the constants $a$ and $b$, such that the function is continuous on the entire real line.
    }\cite{ci}\\
    \begin{enumerate}
        \setcounter{enumi}{4}
        \item $f(x)=\begin{cases}
            3x^2,   &x\geq 1\\
            ax-4,   &x<1
        \end{cases}$
        \item $f(x)=\begin{cases}
            2,      &x\leq -1\\
            ax+b,   &-1<x<3\\
            -2,     &x\geq 3
        \end{cases}$
    \end{enumerate}
\end{exercise}
