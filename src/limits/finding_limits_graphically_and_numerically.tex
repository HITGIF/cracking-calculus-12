\chapterimage{img/inv.jpg} 
\chapter{Finding Limits Graphically and Numerically}

\section{Introduction to Functions}

Many everyday phenomena involve two quantities that are related to 
each other by some rule of correspondence. The mathematical term for 
such a rule of correspondence is a \textbf{relation}. In mathematics, 
relations are often represented by mathematical equations and formulas. 
For instance, the simple interest $I$ earned on \$1000 for 1 year is 
related to the annual interest rate r by the formula $I = 1000r$.

The formula $I = 1000r$ represents a special kind of relation that 
matches each item from one set with \textit{exactly one} item from a 
different set. Such a relation is called a \textbf{function}.

\begin{definition}[Function]
	A \textbf{function} f from a set $A$ to a set $B$ is a relation that 
	assigns to each element $x$ in the set $A$ exactly one element $y$ 
	in the set $B$. The set $A$ is the \textbf{domain} (or set of inputs) 
	of the function $f$, and the set $B$ contains the \textbf{range} 
	(or set of outputs).
\end{definition}

\section{Introduction}
\lipsum[1]