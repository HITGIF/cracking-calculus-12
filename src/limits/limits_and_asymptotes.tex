\chapterimage{img/inf.jpg} 
\chapter{Limits and Asymptotes}

The last thing you need to know about limits is its relationship with the asymptotes on the graph of a function, which is usually tested upon (advise: memorize the propositions for quizzes).

\section{Essential Discontinuities and Vertical Asymptotes}
As aforementioned in the previous chapter, the essential discontinuity, by Definition \ref{essential discontinuity}, occurs when the curve has a vertical asymptote.

Consider the function
$$
f(x) = \frac{1}{(x+1)^2}
$$
While the $\Lim{x\to -1} f(x)$ does not exist, see
Figure~\ref{plot:1/(x+1)^2}, something can still be said. \cite{mooc}
\begin{figure}[H]
    \centering
    \begin{tikzpicture}
        \begin{axis}[
                domain=-2:1,
                ymax=100,
                samples=100,
                axis lines =middle, xlabel=$x$, ylabel=$y$,
                every axis y label/.style={at=(current axis.above origin),anchor=south},
                every axis x label/.style={at=(current axis.right of origin),anchor=west}
            ]
        \addplot [very thick, penColor, smooth, domain=(-2:-1.1)] {1/(x+1)^2};
            \addplot [very thick, penColor, smooth, domain=(-.9:1)] {1/(x+1)^2};
            \addplot [textColor, dashed] plot coordinates {(-1,0) (-1,100)};
            \end{axis}
    \end{tikzpicture}
    \caption{A plot of $f(x)=\protect\frac{1}{(x+1)^2}$. \cite{mooc}}
    \label{plot:1/(x+1)^2}
\end{figure}


\begin{definition}[Infinite Limit]\label{def:inflimit}\index{limit!infinite}\index{infinite limit}
If $f(x)$ grows arbitrarily large as $x$ approaches $a$, we write
\[
\Lim{x\to a} f(x) = \infty
\]
and say that the limit of $f(x)$ \textbf{approaches infinity} as $x$
goes to $a$.

If $|f(x)|$ grows arbitrarily large as $x$ approaches $a$ and $f(x)$ is
negative, we write
\[
\Lim{x\to a} f(x) = -\infty
\]
and say that the limit of $f(x)$ \textbf{approaches negative infinity}
as $x$ goes to $a$.
\\\cite{mooc}
\end{definition}

On the other hand, if we consider the function 
\[
f(x) = \frac{1}{(x-1)}
\]
While we have $\Lim{x\to 1} f(x) \ne \pm\infty$, we do have one-sided
limits, $\Lim{x\to 1+} f(x) = \infty$ and $\Lim{x\to 1-} f(x) =
-\infty$, see Figure~\ref{plot:1/(x-1)}. \cite{mooc}

\begin{figure}[H]
    \centering
    \begin{tikzpicture}
        \begin{axis}[
                domain=-1:2,
                ymax=50,
                ymin=-50,
                samples=100,
                axis lines =middle, xlabel=$x$, ylabel=$y$,
                every axis y label/.style={at=(current axis.above origin),anchor=south},
                every axis x label/.style={at=(current axis.right of origin),anchor=west}
              ]
          \addplot [very thick, penColor, smooth, domain=(1.02:2)] {1/(x-1)};
              \addplot [very thick, penColor, smooth, domain=(-1:.98)] {1/(x-1)};
              \addplot [textColor, dashed] plot coordinates {(1,-50) (1,50)};
            \end{axis}
    \end{tikzpicture}
    \caption{A plot of $f(x)=\protect\frac{1}{x-1}$. \cite{mooc}}
    \label{plot:1/(x-1)}
\end{figure}


\begin{definition}[Vertical Asymptote]\label{def:vert asymptote}\index{asymptote!vertical}\index{vertical asymptote}
If 
\[
\Lim{x\to a} f(x) = \pm\infty, \qquad \Lim{x\to a+} f(x) = \pm\infty, \qquad\text{or}\qquad \Lim{x\to a-} f(x) = \pm\infty,
\]
then the line $x=a$ is a \textbf{vertical asymptote} of $f(x)$.
\\\cite{mooc}
\end{definition}

\begin{example}
    Find the vertical asymptotes of 
    \[
    f(x) = \frac{x^2-9x+14}{x^2-5x+6}.
    \]
    \begin{solution}
        Start by factoring both the numerator and the denominator:
        \[
        \frac{x^2-9x+14}{x^2-5x+6} = \frac{(x-2)(x-7)}{(x-2)(x-3)}
        \]
        Using limits, we must investigate when $x\to 2$ and $x\to 3$. Write
        \begin{align*}
        \Lim{x\to 2} \frac{(x-2)(x-7)}{(x-2)(x-3)} &= \Lim{x\to 2} \frac{(x-7)}{(x-3)}\\
        &= \frac{-5}{-1}\\
        &=5.
        \end{align*}
        Now write
        \begin{align*}
        \Lim{x\to 3} \frac{(x-2)(x-7)}{(x-2)(x-3)} &= \Lim{x\to 3} \frac{(x-7)}{(x-3)}\\
        &= \Lim{x\to 3}\frac{-4}{x-3}.
        \end{align*}
        Since $\Lim{x\to 3+} x-3$ approaches $0$ from the right and the
        numerator is negative, $\Lim{x\to 3+} f(x) = -\infty$. Since
        $\Lim{x\to 3-} x-3$ approaches $0$ from the left and the numerator is
        negative, $\Lim{x\to 3-} f(x) = \infty$. Hence we have a vertical
        asymptote at $x=3$, see Figure~\ref{plot:(x^2-9x+14)/(x^2-5x+6)}.
    \end{solution}
    \begin{figure}[H]
        \centering
        \begin{tikzpicture}
            \begin{axis}[
                    domain=1:4,
                    ymax=50,
                    ymin=-50,
                    samples=100,
                    axis lines =middle, xlabel=$x$, ylabel=$y$,
                    every axis y label/.style={at=(current axis.above origin),anchor=south},
                    every axis x label/.style={at=(current axis.right of origin),anchor=west}
                  ]
              \addplot [very thick, penColor, smooth, domain=(3.02:4)] {(x-7)/(x-3)};
                  \addplot [very thick, penColor, smooth, domain=(1:2.98)] {(x-7)/(x-3)};
                  \addplot [textColor, dashed] plot coordinates {(3,-50) (3,50)};
                  \addplot[color=penColor,fill=background,only marks,mark=*] coordinates{(2,5)};  %% open hole
                \end{axis}
        \end{tikzpicture}
        \caption{A plot of $f(x)=\protect\frac{x^2-9x+14}{x^2-5+6}$. \cite{mooc}}
        \label{plot:(x^2-9x+14)/(x^2-5x+6)}
    \end{figure}
\end{example}

\clearpage
\section{Limits at Infinity and Horizontal Asymptotes}

Consider the function:
$$
f(x) = \frac{6x-9}{x-1}
$$
\begin{figure}[H]
    \centering
    \begin{tikzpicture}
        \begin{axis}[
                domain=1:4,
                ymax=20,
                ymin=-10,
                samples=100,
                axis lines =middle, xlabel=$x$, ylabel=$y$,
                every axis y label/.style={at=(current axis.above origin),anchor=south},
                every axis x label/.style={at=(current axis.right of origin),anchor=west}
            ]
        \addplot [very thick, penColor, smooth, domain=(0:.9)] {(6*x-9)/(x-1)};
            \addplot [very thick, penColor, smooth, domain=(1.1:4)] {(6*x-9)/(x-1)};
            \addplot [textColor, dashed] plot coordinates {(1,-10) (1,20)};
            \addplot [textColor, dashed] plot coordinates {(0,6) (4,6)};
            \end{axis}
    \end{tikzpicture}
    \caption{A plot of $f(x)=\protect\frac{6x-9}{x-1}$. \cite{mooc}}
    \label{plot:(6x-9)/(x-1)}
\end{figure}

As $x$ approaches infinity, it seems like $f(x)$ approaches a specific
value. This is a \textit{limit at infinity}. \cite{mooc}

\begin{definition}[Limit At Infinity]\label{def:limitAtInfty}\index{limit!at infinity}
If $f(x)$ becomes arbitrarily close to a specific value $L$ by making
$x$ sufficiently large, we write
\[
\Lim{x\to \infty} f(x) = L
\]
and we say, the \textbf{limit at infinity} of $f(x)$ is $L$.  

If $f(x)$ becomes
arbitrarily close to a specific value $L$ by making $x$ sufficiently
large and negative, we write
\[
\Lim{x\to -\infty} f(x) = L
\]
and we say, the \textbf{limit at negative infinity} of $f(x)$ is $L$.  
\\\cite{mooc}
\end{definition}

You might have guessed it, this results in a \B{horizontal asymptote}.

\begin{definition}[Horizontal Asymptote]\label{def:horiz asymptote}\index{asymptote!horizontal}\index{horizontal asymptote}
    If  
    \[
    \Lim{x\to \infty} f(x) = L \qquad\text{or}\qquad \Lim{x\to -\infty} f(x) = L,
    \]
    then the line $y=L$ is a \textbf{horizontal asymptote} of $f(x)$.
    \\\cite{mooc}
\end{definition}

\begin{example} 
    Give the horizontal asymptotes of
    \[
    f(x) = \frac{6x-9}{x-1}
    \]
    \cite{mooc}
    \begin{solution}
    From our previous work, we see that $\Lim{x\to \infty} f(x) = 6$, and
    upon further inspection, we see that $\Lim{x\to -\infty} f(x) =
    6$. Hence the horizontal asymptote of $f(x)$ is the line $y=6$.
    \end{solution}
\end{example}

\begin{exercise}
    ~\\\-\hspace{0.3cm} \textbf{
        In Exercises 1–2, find the vertical asymptote(s).
    }\cite{mooc}\\
    \begin{enumerate} 
		\item $f(x) = \displaystyle\frac{x-3}{x^2+2x-3}.$
		\item $f(x) = \displaystyle\frac{x^2-x-6}{x+4}.$
    \end{enumerate}
    ~\\\-\hspace{0.3cm} \textbf{
        In Exercises 3–4, find the horizontal asymptote(s).
    }\cite{mooc}\\
    \begin{enumerate}
        \setcounter{enumi}{2}
        \item $f(x) = \displaystyle\frac{\sin\left(x^7\right)}{\sqrt{x}}$
        \item $f(x) = \displaystyle\left(17 + \frac{32}{x} - \frac{\left(\sin(x/2)\right)^2}{x^3}\right)$
        \item Suppose a population of feral cats on a certain college campus $t$
        years from now is approximated by
        $$
        p(t) = \frac{1000}{5+ 2e^{-0.1 t}}.
        $$
        Approximately how many feral cats are on campus 10 years from now? 50 years from now? 100 years from now? 1000 years from now? What do you notice about the prediction---is this realistic? \cite{mooc}
        \item The amplitude of an oscillating spring is given by
        $$
        a(t) = \frac{\sin(t)}{t}.
        $$
        What happens to the amplitude of the oscillation over a long period of time? \cite{mooc}
    \end{enumerate}
\end{exercise}
