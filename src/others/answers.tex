\chapterimage{img/lim_intro.jpg} 
\chapter*{Answers to Chapter Exercises}
\markboth{{Answers to Chapter Exercises}}{Answers to Chapter Exercises}
\addcontentsline{toc}{chapter}{\textcolor{black}{Answers to Chapter Exercises}}

\section*{Chapter 1}
\twocol
\begin{enumerate}
    \item domain; range; function
    \item Verbally; Numerically; Graphically; Analytically
    \item independent; dependent
    \item piecewise
    \item (a) -1 (b) -9 (c) $2x-5$
    \item (a) -7 (b) 4 (c) 9
    \item (a) 19 (b) 17 (c) 0
\end{enumerate}
\endtwocol

\section*{Chapter 2}
\twocol
\begin{enumerate}
    \item (a) $\pm 6$ (b) Odd multiplicity; number of turning points: 1
    \item (a) $0, 2\pm \sqrt{3}$ (b) Odd multiplicity; number of turning points: 2
    \item (a) $\pm 2, -3$ (b) Odd multiplicity; number of turning points: 2
    \item (a) $0, \pm \sqrt{3}$ (b) 0, odd multiplicity; $\pm \sqrt{3},$ even multiplicity; number of turning points: 4
    \item $x^2-8x$
    \item $x^4-4x^3-9x^2+36x$
    \item $x^2-2x-2$
    \item $x^3+9x^2+20x$
    \item $x=-2$; $y=0$
    \item $x=2, x=\pm 1$; $y=0$
    \item $x=2$; $y=1$
    \item None; None
\end{enumerate}
\endtwocol

\section*{Chapter 3}
\twocol
\begin{enumerate}
    \item (a) $x^2+4x-5$ (b) $x^2-4x+5$ (c) $4x^3-5x^2$ (d) $\D\dfrac{x^2}{4x-5}$
    \item (a) $\D\dfrac{x+1}{x^2}$ (b) $\D\dfrac{x-1}{x^2}$ (c) $\D\dfrac{1}{x^3}$
    \item 3
    \item 74
    \item $\dfrac{3}{5}$
\end{enumerate}
\endtwocol

\section*{Chapter 4}
\twocol
\begin{enumerate}
    \item $f(g(x))=f(x/2)=2(x/2)=x$; $g(f(x))=g(2x)=(2x)/2=x$
    \item $f(g(x))=f(1/x)=1/(1/x)=x$; $g(f(x))=g(1/x)=1/(1/x)=x$
    \item $g^{-1}(x)=8x$
    \item No inverse
    \item $f^{-1}(x)=\D\dfrac{x^2-3}{2}$
    \item No inverse
\end{enumerate}
\endtwocol

\section*{Chapter 5}
\twocol
\begin{enumerate}
    \item Let $\epsilon >0$. Set $\delta = \epsilon$. If $0<|x-0| <\delta$,
    then $|x\cdot 1|<\epsilon$, since $\sin\left(\frac{1}{x}\right)\le
    1$, $|x\sin \left( {1\over x}\right)-0|< \epsilon$.
    \item Let $\epsilon > 0$.  No matter what I choose for $\delta$, if $x$ is within $\delta$ of $-2$, then $\pi$ is within $\epsilon$ of $\pi$.
    \item Let $\epsilon > 0$.  Set $\delta = 3\epsilon$.  Assume $0 < |x-9| < \delta$.  Divide both sides by $3$ to get $\frac{|x-9|}{3} < \epsilon$.  Note that $\sqrt{x}+3 > 3$, so $\frac{|x-9|}{\sqrt{x} + 3} < \epsilon$.  This can be rearranged to conclude $\left|\frac{x-9}{\sqrt{x} - 3} - 6\right| < \epsilon$.
    \item Consider what happens when $x$ is near zero and positive, as compared to when $x$ is near zero and negative.
\end{enumerate}
\endtwocol
\clearpage
\section*{Chapter 6}
\twocol
\begin{enumerate}
    \item 8
    \item 1/5
    \item -1/9
    \item 1/6
    \item $3x^2$
    \item 4
    \item b
\end{enumerate}
\endtwocol

\section*{Chapter 7}
\twocol
\begin{enumerate}
    \item Continuous for all real $x$
    \item Nonremovable discontinuity at $x=1$\\Removable discontinuity at $x=0$
    \item Removable discontinuity at $x=-2$\\Nonremovable discontinuity at $x=5$
    \item Nonremovable discontinuity at $x=-7$
    \item $a=7$
    \item $a=-1,b=1$
\end{enumerate}
\endtwocol

\section*{Chapter 8}
\twocol
\begin{enumerate}
    \item $x = 1$ and $x = -3$
    \item $x = -4$
    \item $y=0$
    \item $y=17$
    \item After 10 years, $\approx 174$ cats; after 50 years, $\approx 199$ cats; after 100 years, $\approx 200$ cats; after 1000 years, $\approx 200$ cats; in the sense that the population of cats cannot grow indefinitely this is somewhat realistic.
    \item The amplitude goes to zero. 
\end{enumerate}
\endtwocol

\section*{Chapter 9}
\twocol
\begin{enumerate}
    \item $f(2) =  10$ and $f'(2) = 7$
    \item $f'(-2) = 4$
    \item $f(1.2) \approx 2.2$
    \item $(0,4.5)\cup(4.5,6)$ 
    \item $(0,3)\cup(3,4.5)\cup(4.5,6)$
    \item See figure on the right.
    \begin{figure}[H]
        \centering
        \begin{tikzpicture}
          \begin{axis}[
                  domain=0:6,
                  ymax=1.5,
                  ymin=-1.5,
                  samples=100,
                  axis lines =middle, xlabel=$x$, ylabel=$y$,
                  every axis y label/.style={at=(current axis.above origin),anchor=south},
                  every axis x label/.style={at=(current axis.right of origin),anchor=west},
                  grid=both,
                  grid style={dashed, gridColor},
                  xtick={0,...,6},
                  ytick={-1,...,1},
                ]
                \addplot [very thick, penColor, smooth,domain=(0:3)] {-(pi/3)*cos(deg(pi*x/3)))};
                \addplot [very thick, penColor, smooth,domain=(3:6)] {-(pi/3)*cos(deg(pi*(x-3)/3)))};
                \addplot[color=penColor,fill=background,only marks,mark=*] coordinates{(3,pi/3)};  %% open hole
                \addplot[color=penColor,fill=background,only marks,mark=*] coordinates{(3,-pi/3)};  %% open hole
                \addplot[color=penColor,fill=background,only marks,mark=*] coordinates{(4.5,0)};  %% open hole
              \end{axis}
      \end{tikzpicture}
      Answer to Question 6. A sketch of $f'(x)$. \cite{mooc}
      \end{figure}
\end{enumerate}
\endtwocol

\section*{Chapter 10}
\twocol
\begin{enumerate}
    \item 0
    \item 0
    \item $\pi x^{\pi-1}$
    \item $-(9/7)x^{-16/7}$
    \item $15x^2+24x$
    \item $-5x^{-6} - x^{-3/2}/2$
    \item $3x^2(x^3-5x+10)+x^3(3x^2-5)$
    \item $(x^2+5x-3)(5x^4-18x^2+6x-7)+(2x+5)(x^5-6x^3+3x^2-7x+1)$
    \item ${3x^2\over x^3-5x+10}-{x^3(3x^2-5)\over (x^3-5x+10)^2}$
    \item ${2x+5\over x^5-6x^3+3x^2-7x+1}-{(x^2+5x-3)(5x^4-18x^2+6x-7)\over(x^5-6x^3+3x^2-7x+1)^2}$
    \item $6+18x$
    \item ${1\over2}\left({-169\over x^2}-1\right)\Big/\sqrt{{169\over x}-x}$
    \item ${300 x \over(100-x^2)^{5/2}}$
    \item $\left(4x(x^2+1)+{4x^3+4x\over2\sqrt{1+(x^2+1)^2}}\right)\Big/$\hfill\break$2\sqrt{(x^2+1)^2+\sqrt{1+(x^2+1)^2}}$
    \item $6x(2x-4)^3+6(3x^2+1)(2x-4)^2$
    \item $-5/(3x-4)^2$
    \item $56x^6+72x^5+110x^4+100x^3+60x^2+28x+6$
    \item $4x^3-9x^2+x+7$
\end{enumerate}
\endtwocol

\section*{Chapter 11}
\twocol
\begin{enumerate}
    \item $-x/y$
    \item $-(2x+y)/(x+2y)$
    \item $(2xy-3x^2-y^2)/(2xy-3y^2-x^2)$
    \item $-\sqrt{y}/\sqrt{x}$
    \item $\frac{y^{3/2}-2}{1-y^{1/2}3x/2}$
    \item $-y^2/x^2$
    \item $-4/y^3$
    \item $(3x)/(4y)$
\end{enumerate}
\endtwocol

\section*{Chapter 12}
\twocol
\begin{enumerate}
    \item $c=\dfrac{6\pm \sqrt{3}}{3}$
    \item Not differentiable at $x=0$
    \item $c=-2+\sqrt{5}$
    \item $c\-1/2$
    \item $c=\pm 1/\sqrt{3}$
    \item $f$ is not differentiable at $x=-\dfrac{1}{2}$
\end{enumerate}
\endtwocol

\section*{Chapter 13}
\twocol
\begin{enumerate}
    \item min at $x=1/2$
    \item min at $x=-1$, max at $x=1$
    \item max at $x=2$, min at $x=4$
    \item min at $x=\pm 1$, max at $x=0$.
    \item min at $x=0$, max at $x=\frac{3\pm \sqrt{17}}{2}$
    \item none
    \item relative min at $x=0$
\end{enumerate}
\endtwocol

\section*{Chapter 14}
\twocol
\begin{enumerate}
    \item min at $x=1/2$
    \item min at $x=-1$, max at $x=1$
    \item max at $x=2$, min at $x=4$
    \item max at $x=0$, min at $x=\pm 11$
    \item $f'(x) = 2ax + b$, this has only one root and hence one critical point; $a<0$ to guarantee a maximum.
    \item $y$-intercept at $(0,0)$; no vertical asymptotes; critical points:
    $x=\pm1/\sqrt[4]{5}$; local max at $x=-1/\sqrt[4]{5}$, local min at
    $x=-1/\sqrt[4]{5}$; increasing on $(-\infty,-1/\sqrt[4]{5})$, decreasing
    on $(-1/\sqrt[4]{5},1/\sqrt[4]{5})$, increasing on
    $(1/\sqrt[4]{5},\infty)$; concave down on $(-\infty,0)$, concave up on
    $(0, \infty)$; root at $x=0$; no horizontal asymptotes; interval for
    sketch: $[-1.2,1.2]$ (answers may vary)
    \item $y$-intercept at $(0,0)$; no vertical asymptotes; critical points: $x=
    1$; local max at $x=1$; increasing on $[0,1)$, decreasing on
      $(1,\infty)$; concave down on $[0,\infty)$; roots at $x=0$, $x=4$;
        no horizontal asymptotes; interval for sketch: $[0,6]$ (answers
        may vary)
    \item $y$-intercept at $(0,0)$; no vertical asymptotes; critical points:
    $x=0$, $x=1$, $x=3$; local max at $x=1$, local min at $x=3$;
    increasing on $(-\infty,0)$ and $(0,1)$, decreasing on $(1,3)$,
    increasing on $(3,\infty)$; concave down on $(-\infty,0)$, concave
    up on $(0, (3-\sqrt{3})/2)$, concave down on
    $((3-\sqrt{3})/2,(3+\sqrt{3})/2)$, concave up on
    $((3+\sqrt{3})/2,\infty)$; roots at $x=0$, $x= \frac{5\pm
      \sqrt{5}}{2}$; no horizontal asymptotes; interval for sketch:
    $[-1,4]$ (answers may vary)
    \item no $y$-intercept; vertical asymptote at $x=0$; critical points: $x=0$,
    $x=\frac{1}{\sqrt[3]{2}}$; local min at $x=\frac{1}{\sqrt[3]{2}}$;
    decreasing on $(-\infty,0)$, decreasing on
    $(0,\frac{1}{\sqrt[3]{2}})$, increasing on
    $(\frac{1}{\sqrt[3]{2}},\infty)$; concave up on $(-\infty,-1)$,
    concave down on $(-1,0)$, concave up on $(0,\infty)$; root at $x=-1$;
    no horizontal asymptotes; interval for sketch: $[-3,2]$ (answers may
    vary)
\end{enumerate}
\endtwocol

\section*{Chapter 15}
\twocol
\begin{enumerate}
    \item $20/(3\pi)$ cm/s
    \item $5\sqrt{10}/2$ m/s
    \item $1/4$ m/s
    \item tip: 6 ft/s, length: $5/2$ ft/s
    \item $3/256$ m/s$^2$
    \item $-4.9t^2+100,\qquad\text{(ignoring air resistance)}$
\end{enumerate}
\endtwocol

\section*{Chapter 16}
\twocol
\begin{enumerate}
    \item $25\times 25$
    \item $w=l=2\cdot 5^{2/3}$, $h=5^{2/3}$, $h/w=1/2$
    \item $w=l=2^{1/3}V^{1/3}$, $h=V^{1/3}/2^{2/3}$, $h/w=1/2$
    \item \$5000
    \item Go direct from $A$ to $D$.
    \item $h/r=2$
    \item $4/27$
    \item $P$ should be at distance $c\root 3\of {a} /
    (\root 3\of {a} + \root 3\of {b})$ from charge $A$.
    \item The ratio of the volume of the sphere to the volume of the
    cone is $1033/4096+33/4096\sqrt{17}\approx 0.2854$, so the cone
    occupies approximately $28.54\%$ of the sphere.
\end{enumerate}
\endtwocol
